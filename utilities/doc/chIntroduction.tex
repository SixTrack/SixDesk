\chapter{Introduction} \label{Intro}
\SIXTRACK{}~\cite{SixTrack_user_manual,SixPub,sixtrackWeb} is a tracking
code for simulating transverse and longitudinal single particle beam dynamics.
Tracking is treated in a full six--dimensional way, including synchrotron
motion, in a symplectic manner. \SIXTRACK{} is widely used at CERN for
predicting dynamic aperture in large storage
rings~\cite{DynApeStudiesGiovannozzi2015} like the Large Hadron Collider
(LHC)~\cite{NomLHCdesignRepoV1} or its upgrade as foreseen by the
High Luminosity LHC Project (HL-LHC)~\cite{HLLHC_book,HLLHCtechDesRepo}.

The code was extended~\cite{SixTrackForCollimation} to predict the
performance of a collimation system in terms of loss pattern and cleaning
inefficiency. Hence, \SIXTRACK{} is routinely used nowadays also
for addressing the performance of existing cleaning systems,
like those of the LHC~\cite{LHCCollSys} or of the Relativistic
Heavy Ion Collider (RHIC) at BNL~\cite{RHICcollSys}, or new ones.

The code is in continuous development~\cite{HLLHCTrackWS,Amereghe6TColl},
not only to improve the accuracy of the tracking models, but also including
the dynamics introduced by novel accelerator technologies, like electron
lenses or powered wires for the compensation of beam--beam long range effects
or christal collimation.

The accelerator dynamic aperture is studied scanning
the beam phase space in presence of non-linear forces, like the kicks
introduced by long range beam--beam interactions or multipolar components
of magnetic fields. Moreover, the scan could be also performed varying
the machine configurations. The
\SIXDESK{}~\cite{SixDesk_original,SixDesk_updated} environment gives the
users of \SIXTRACK{} a mean to handle the large amount of files to be treated.

\section{Assumed Environment}\label{Sec:Env}
\begin{table}[h]
\begin{center}
    \caption{Environment Variables.}
    \label{tab:EnvVarsExplain}
    \begin{tabular}{|l|l|}
    \hline
    \rowcolor{blue!30}
    \textbf{Variable Name} & \textbf{Value} \\
    \hline
    \texttt{appNameDef} & \texttt{\whichSixTrack{}} \\
    \hline
    \texttt{newBuildPathDef} & \texttt{/afs/cern.ch/project/sixtrack/build} \\
    \hline
    \texttt{SixDeskTools} & \texttt{/afs/cern.ch/project/sixtrack/SixDesk\_utilities/dev} \\
    \hline
    \end{tabular}
\end{center}
\end{table}
Throughout the manual, the environment in \texttt{lxplus.cern.ch},
native to \SIXDESK{}, will be assumed and the
environment variables listed in Tab.~\ref{tab:EnvVarsExplain} will be
considered; these are automatically set by \SIXDESK{}.
Since \SIXDESK{} is native to , the user has nothing
to set up their environment.
\TODO{What to do in case of local installations or installations on other
clusters?}

\begin{table}[h]
\begin{center}
    \caption{Pre-Requisites}
    \label{tab:Pre-Requisites}
    \begin{tabular}{|l|l|}
    \hline
    \rowcolor{blue!30}
    \textbf{Component} & \textbf{reason} \\
    \hline
    kerberos & to renew/check credentials via \texttt{klist} and \texttt{kinit} \\
    \hline
    AFS (local mount) & retrieval of optics files \\
    & submission to BOINC via spooldir\\
    \hline
    HTCondor (local installation) & submission of jobs to local batch system \\
    \hline
    \texttt{python2.7} & \texttt{SixDB} \\
    & computation of floating point scan parameters \\
    \hline
    \end{tabular}
\end{center}
\end{table}

\section{Overview} \label{Overview}
\TODO{Logics behind DA scans:}
\begin{enumerate}
\item prepare the \emph{input files}, i.e.~\texttt{sixdeskenv},
\texttt{sysenv} and \texttt{fort.3.local}
\item generate files describing the \emph{accelerator geometry}
  with \MADX{} (\texttt{fort.2}, \texttt{fort.8},
  \texttt{fort.16}); then, run \SIXTRACK{}; then, collect results
  (\texttt{fort.10}) and analyse them via \SIXDB{};
\item inner loops (i.e.~controlled by \texttt{sixdeskenv}) and outer loops
  (i.e.~controlled by \texttt{scan\_definitions});
\end{enumerate}

\section{Work Flow}
Show workflow of production of results, both for BOINC (including ``processed''
folder) and HTCondor.

Retrieval of results depends on the submission platform:
\begin{itemize}
\item \texttt{run\_results}: BOINC
\item \texttt{run\_status}: HTCondor, HTBoinc
\end{itemize}

\section{Scans}\label{Sec:NativeScans}
The scans performed by \SIXDESK{} (so-called ``native'') allow to estimate
the dynamic aperture for a given machine configuration, mainly probing the
beam phase space via a linear scan in particle amplitude parametric in
angle. These scans cover different error configurations of the magnetic
fields, and optionally, the user can also request
to replicate the study varying the machine tune.
Scans are handled by \SIXDESK{} with the input coded
in the \texttt{sixdeskenv} file.
Table~\ref{tab:InternalScanParamters} summarises essential technical
characteristics of the \SIXDESK{} ``native'' scans.
\begin{table}[t]
\begin{center}
    \caption{Essential technical
      characteristics of the scans native to \SIXDESK{}.}
    \label{tab:InternalScanParamters}
    \begin{tabular}{|c|l|l|}
    \hline
    \rowcolor{blue!30}
    \textbf{Category} & \textbf{Variable} & \textbf{Comment} \\
    \hline
    \multirowcell{2}{beam \\ phase space}
    & amplitude & main loop in \SIXDESK{}, sub-loop in \SIXTRACK{} \\
    \cline{2-3}
    & angle     & loop in \SIXDESK{}, set point in \SIXTRACK{} \\
    \hline
    \multirowcell{2}{machine \\ phase space}
    & magnetic errors (seed) & loop in \SIXDESK{}, a \MADX{} job each\\
    \cline{2-3}
    & tune & loop in \SIXDESK{}, each \SIXTRACK{} job matches the tune \\
    \hline
    \end{tabular}
\end{center}
\end{table}

A \SIXDESK{} study is exactly made of a complete ``native'' scan, with all the
\SIXTRACK{} input files describing the machine (see Sec.~\ref{Overview})
generated by a single \texttt{*.mask} file. The beam phase space is scanned
based on the settings in \texttt{sixdeskenv} file, and machine parameters like
the multipolar errors and the tune are treated as ``close'' variations of the
original study case.

``Non-native'' scans are available as well, to extend the set of scans
that can be performed (please see Sec.~\ref{ExternalScans}).

\section{Input Files}
\label{Sec:InputFiles}
\begin{table}[h]
\begin{center}
    \caption{Regular SixTrack input files for DA scans.}
    \label{tab:6TinpFiles}
    \begin{tabular}{|l|l|}
    \hline
    \rowcolor{blue!30}
    \textbf{File} & \textbf{Description} \\
    \hline
    \texttt{fort.2} & machine lattice and the nominal powering of magnets \\
    \hline
    \texttt{fort.3} & global simulation parameters and definition of special elements \\
    \hline
    \texttt{fort.8} & misalignments and tilt angles to be assigned to machine elements \\
    \hline
    \texttt{fort.16} & multipole errors to be assigned to magnetic elements \\
    \hline
    \end{tabular}
\end{center}
\end{table}
In order to perform a Dynamic Aperture (DA) study, SixTrack must be
provided with three geometry files~\cite{SixTrack_user_manual},
i.e.~files that describe the geometry and settings of the machine lattice.
A further input file, i.e.~\texttt{fort.3}~\cite{SixTrack_user_manual},
is needed to set global simulation parameters of the SixTrack job and
control special elements (e.g.~beam-beam elements, wires,
e-lenses, etc\ldots).
A summary table is found in Tab.~\ref{tab:6TinpFiles}.

\begin{table}[h]
\begin{center}
    \caption{Input files for DA (inner) scans.}
    \label{tab:InnerScanInputFiles}
    \begin{tabular}{|l|l|}
    \hline
    \rowcolor{blue!30}
    \textbf{File} & \textbf{Description} \\
    \hline
    \texttt{sixdeskenv} & main simulation parameters, scan ranges and some environment parameters \\
    \hline
    \texttt{sysenv} & additional environment variables \\
    \hline
    \texttt{fort.3.local} & additional parameters for each SixTrack job \\
    \hline
    \end{tabular}
\end{center}
\end{table}
Internal scans of DA (see Sec.~\ref{Overview}) are controlled by means of two input
files, i.e.~\texttt{sideskenv} and \texttt{sysenv}. A third input file,
\texttt{fort.3.local}, can be used to add simulation parameters to
the \texttt{fort.3} file. This file is optional.
A summary table is found in Tab.~\ref{tab:InnerScanInputFiles}.
Outer scans of DA (see Sec.~\ref{Overview}) are controlled by further input files
(see Sec.~\ref{ExternalScans}).

\subsection{\texttt{sixdeskenv}}\label{Sec:InputFiles:sixdeskenv}
This file contains main simulation parameters, scan ranges and some environment
parameters.
They are summarised in Tab.~\ref{tab:sixdeskenv}.
\begin{table}[h]
\begin{center}
    \caption{User-defined parameters of the \texttt{sixdeskenv} file.}
    \label{tab:sixdeskenv}
    \begin{tabular}{|p{5cm}|p{10cm}|}
    \hline
    \rowcolor{blue!30}
    \textbf{Name} & \textbf{Description} \\
    \hline
    \texttt{additionalFilesOutMAD} & list of \MADX{} output files that
    should be fed into \SIXTRACK{} jobs in addition to the geometry ones
    (see Sec.~\ref{Sec:InputFiles}). Please leave blank if not needed;
    otherwise, please list all filenames (no paths) separate by
    whitespaces; as it happens for the geometry ones, one file per seed
    will be generated and gzipped. See also Sec.~\ref{Sec:AdditionalFiles}. \\
    \multicolumn{2}{|r|}{e.g.~\texttt{export additionalFilesOutMAD="fc.3.aper additional.txt"} } \\
    \hline
    \texttt{additionalFilesInp6T} & list of additional \SIXTRACK{} input files.
    Please leave blank if not needed; otherwise, please list all filenames
    (no paths) separate by whitespaces; every file will be used as it is
    by all \SIXTRACK{} jobs. The files must be present either in the
    \texttt{sixjobs} directory or in the respective study subfolder of
    the \texttt{studies} directory. See also Sec.~\ref{Sec:AdditionalFiles}. \\
    \multicolumn{2}{|r|}{e.g.~\texttt{export additionalFilesInp6T="elens1.dat elens2.dat"} } \\
    \hline
    \end{tabular}
\end{center}
\end{table}

\subsection{\texttt{sysenv}}\label{Sec:InputFiles:sysenv}
This file contains additional environment variables.
They are summarised in Tab.~\ref{tab:sysenv}.
\begin{table}[h]
\begin{center}
    \caption{User-defined parameters of the \texttt{sysenv} file.}
    \label{tab:sysenv}
    \begin{tabular}{|p{3cm}|p{12cm}|}
    \hline
    \rowcolor{blue!30}
    \textbf{Name} & \textbf{Description} \\
    \hline
    \texttt{appName} & name of the executable.
    If left blank, it defaults to \texttt{\whichSixTrackVersion{}} (see Sec.~\ref{Sec:SixTrackExes}).
    It is mandatory if the user wants to specify \texttt{appVer}. \\
    & e.g.~\texttt{export appName=\whichSixTrack{}} \\
    \hline
    \texttt{appVer} & version of the executable.
    If left blank, it defaults to latest version (see Sec.~\ref{Sec:SixTrackExes}).
    If not left blank, \texttt{appName} must be set either. \\
    & e.g.~\texttt{export appVer=\whichSixTrackVersion{}} \\
    \hline
    \texttt{SIXTRACKEXE} & full path to sixtrack executable.
    Used only by HTCondor and single turn jobs, ignored by BOINC
    (see Sec.~\ref{Sec:SixTrackExes}). \\
    & e.g.~\texttt{export SIXTRACKEXE=\$sixdeskpath/exes/\$appName} \\
    \hline
    \end{tabular}
\end{center}
\end{table}

\subsubsection{SixTrack Executable and its Full Path}\label{Sec:SixTrackExes}
In case of a job submitted to BOINC, a volunteer receives
a copy of the executable together with the input files. In order to
be trustable by volunteers, the executable must be signed.
Hence, no custom-made executable can be sent to volunteers, and
only signed executables (prepared on purpose by the admins)
are made available to users.
Therefore, in case of running jobs on BOINC, it is important
that the user specifies which application to use and which version,
via the \texttt{appName} and \texttt{appVer} variable. If left
blank, SixDesk will set automatically the two in order to have
the latest version of SixTrack / SixTrackTest.

The \texttt{appName} and \texttt{appVer} variables can be used
also for single turn jobs (run on the login node on lxplus)
and for jobs run on HTCondor; in this way, no matter the platform,
the user deals with the same interface. Nevertheless,
the user is given the possibility to specify the full path to
the requested SixTrack executable, but, as already mentioned, the
path will be used only for single turn jobs and HTCondor jobs and
\emph{ignored} in case of BOINC.

Therefore, it is recommended to \emph{not define at all}
the variable \texttt{SIXTRACKEXE} and to \emph{leave blank}
the variables \texttt{appName} and \texttt{appVer} and let
SixDesk define them. In this case, SixDesk will set
\begin{lstlisting}
SIXTRACKEXE = ${newBuildPathDef}/${appNameDef}
\end{lstlisting}
If the user defines \texttt{appName}, then SixDesk will set
\begin{lstlisting}
SIXTRACKEXE = ${newBuildPathDef}/${appName}
\end{lstlisting}
If the user also defines \texttt{appVer}, then SixDesk will set
\begin{lstlisting}
SIXTRACKEXE = ${newBuildPathDef}/${appVer}/${appName}
\end{lstlisting}
The existence of \texttt{\${SIXTRACKEXE}} is anyway checked.
For \texttt{newBuildPathDef} and \texttt{appNameDef},
please see Sec.~\ref{Sec:Env}.

\section{The BOINC Platform for Volunteering Computing}
BOINC vs local batch system (e.g.~HTCondor)

