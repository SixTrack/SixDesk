\chapter{Introduction} \label{Intro}
\SIXTRACK{}~\cite{SixTrack_user_manual,SixPub,sixtrackWeb} is a tracking
code for simulating transverse and longitudinal single particle beam dynamics.
Tracking is treated in a full six--dimensional way, including synchrotron
motion, in a symplectic manner. \SIXTRACK{} is widely used at CERN for
predicting dynamic aperture in large storage
rings~\cite{DynApeStudiesGiovannozzi2015} like the Large Hadron Collider
(LHC)~\cite{NomLHCdesignRepoV1} or its upgrade as foreseen by the
High Luminosity LHC Project (HL-LHC)~\cite{HLLHC_book,HLLHCtechDesRepo}.

The code was extended~\cite{SixTrackForCollimation} to predict the
performance of a collimation system in terms of loss pattern and cleaning
inefficiency. Hence, \SIXTRACK{} is routinely used nowadays also
for addressing the performance of existing cleaning systems,
like those of the LHC~\cite{LHCCollSys} or of the Relativistic
Heavy Ion Collider (RHIC) at BNL~\cite{RHICcollSys}, or new ones.

The code is in continuous development~\cite{HLLHCTrackWS,Amereghe6TColl},
not only to improve the accuracy of the tracking models, but also including
the dynamics introduced by novel accelerator technologies, like electron
lenses or powered wires for the compensation of beam--beam long range effects
or christal collimation.

The accelerator dynamic aperture is studied scanning
the beam phase space in presence of non-linear forces, like the kicks
introduced by long range beam--beam interactions or multipolar components
of magnetic fields. Moreover, the scan could be also performed varying
the machine configurations. The
\SIXDESK{}~\cite{SixDesk_original,SixDesk_updated} environment gives the
users of \SIXTRACK{} a mean to handle the large amount of files to be treated.

\section{Overview} \label{Overview}
\begin{enumerate}
\item prepare the \emph{input files}, i.e.~\texttt{sixdeskenv},
\texttt{sysenv} and \texttt{fort.3.local}
\item \SIXTRACK{} generate file describing the \emph{accelerator geometry}
  with \MADX{} (\texttt{fort.2},\texttt{fort.8},
  \texttt{fort.16}); then, run \SIXTRACK{}; then, collect results
  (\texttt{fort.10}) and analyse them via \SIXDB{};
\item inner loops (i.e.~controlled by \texttt{sixdeskenv}) and outer loops
  (i.e.~controlled by \texttt{scan\_definitions});
\end{enumerate}

\section{Work Flow}
Show workflow of production of results, both for BOINC (including ``processed''
folder) and HTCondor.

Retrieval of results depends on the submission platform:
\begin{itemize}
\item \texttt{run\_results}: BOINC
\item \texttt{run\_status}: HTCondor, HTBoinc
\end{itemize}

\section{Input Files}
\begin{description}
\item[\texttt{sixdeskenv}]
\item[\texttt{sysenv}]
\item[\texttt{fort.3.local}]
\end{description}
Geometry files: \texttt{fort.2}, \texttt{fort.8}, \texttt{fort.16}.
    

\section{The BOINC Platform for Volunteering Computing}
BOINC vs local batch system (e.g.~HTCondor)

\section{Pre-requisites}
\SIXDESK{} is native to \texttt{lxplus.cern.ch}. Hence, for running in such
an environment, the user does not need to set up anything. On the contrary,
in case of a local machine or other distributed resources,

\begin{table}[h]
\begin{center}
    \caption{Pre-Requisites}
    \label{tab:Pre-Requisites}
    \begin{tabular}{|l|l|}
    \hline
    \rowcolor{blue!30}
    \textbf{Component} & \textbf{reason} \\
    \hline
    kerberos & to renew/check credentials via \texttt{klist} and \texttt{kinit} \\
    \hline
    AFS (local mount) & retrieval of optics files \\
    & submission to BOINC via spooldir\\
    \hline
    HTCondor (local installation) & submission of jobs to local batch system \\
    \hline
    \texttt{python2.7} & \texttt{SixDB} \\
    & computation of floating point scan parameters \\
    \hline
    \end{tabular}
\end{center}
\end{table}
