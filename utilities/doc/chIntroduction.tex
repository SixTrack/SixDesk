\chapter{Introduction} \label{Intro}

\section{Work Flow}
Show workflow of production of results, both for BOINC (including ``processed''
folder) and HTCondor.

Retrieval of results depends on the submission platform:
\begin{itemize}
\item \texttt{run\_results}: BOINC
\item \texttt{run\_status}: HTCondor, HTBoinc
\end{itemize}

\section{The BOINC Platform for Volunteering Computing}
BOINC vs local batch system (e.g.~HTCondor)

\section{Pre-requisites}
\SIXDESK{} is native to \texttt{lxplus.cern.ch}. Hence, for running in such
an environment, the user does not need to set up anything. On the contrary,
in case of a local machine or other distributed resources,

\begin{table}[h]
\begin{center}
    \caption{Pre-Requisites}
    \label{tab:Pre-Requisites}
    \begin{tabular}{|l|l|}
    \hline
    \rowcolor{blue!30}
    \textbf{Component} & \textbf{reason} \\
    \hline
    kerberos & to renew/check credentials via \texttt{klist} and \texttt{kinit} \\
    \hline
    AFS (local mount) & retrieval of optics files \\
    & submission to BOINC via spooldir\\
    \hline
    HTCondor (local installation) & submission of jobs to local batch system \\
    \hline
    \texttt{python2.7} & \texttt{SixDB} \\
    & computation of floating point scan parameters \\
    \hline
    \end{tabular}
\end{center}
\end{table}
