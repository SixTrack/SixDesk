\chapter{Giudelines and Common Pitfalls} \label{Guidelines}

\section{Naming Convention}
\subsection{Study and Workspace Names}
\begin{itemize}
\item avoid the use of ``\texttt{\_\_}'' (i.e.~a string of two
  consecutive underscore characters), as this is used by the BOINC
  assimilator to properly disentangle study name and name of job in
  the study;
\item avoid the use of the platform in lower case explicitly in the name;
\end{itemize}

\section{Choice of Platform}
HTCondor is convenient when:
\begin{enumerate}
\item results should be collected quickly. This can be the case when
  the user has short time to collect data or the simulation set-up
  is being defined. In the second case, indeed, one does not want to wait
  too long for proceeding;
\item short or few jobs per study. This can be the case when re-submission
  of selected cases is necessary, e.g.~to complete a study when few points
  in the scan are missing;
\end{enumerate}  

The BOINC platform for volunteer computing is convenient in case of
large simulation campaigns, i.e.~when simulations are long or they
are in high number (e.g.~hundreds of thousands of jobs).

Not more than 5 scripts per user running at the same time, for ease
of functionality of afs.
