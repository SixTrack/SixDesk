\begin{titlepage}
\begin{center}\normalsize\scshape
    European Organization for Nuclear Research \\
    CERN BE/ABP
\end{center}
\vspace*{2mm}
\begin{flushright}
    CERN/xx/xx \\
    Updated April 2018
\end{flushright}
\begin{center}\Huge
    \textbf{SixDesk} \\
    \LARGE Version 1.0 \\
    \vspace*{8mm} the Simulation Environment for SixTrack\\
    \vspace*{8mm}\textbf{User's Reference Manual}
\end{center}
\begin{center}
  R.~De Maria, M.~Giovannozzi, E.~McIntosh, A.~Mereghetti, F.~Schmidt,
  I.~Zacharov \\
    \vspace*{4mm}Updated by:
    P.~D.~Hermes, D.~Pellegrini, S.~Kostoglou
\end{center}
\begin{center}\large
    \vspace*{10mm}\textbf{Abstract} \\
\end{center}
\SIXTRACK{}~\cite{SixTrack_user_manual} is a single particle tracking
code widely used at CERN. One of its
most important applications is the estimation of the dynamic aperture available
in large storage rings like the Large Hadron Collider (LHC) or the Future
Circular Collider (FCC). These studies require massive computing resources,
since they consist of scans over large parameter spaces probing non-linear beam
dynamics over long times.
The \SIXDESK{}~\cite{SixDesk_original,SixDesk_updated} environment is the
simulation framework used to manage and control the large amount of
information necessary for and produced by the studies. \\
This document updates the previous documentation, and describes how massive
tracking campaigns can be performed with \SIXTRACK{}
starting from a \MADX{} ``mask'' file.
The \SIXDESK{} environment is an ensemble of shell scripts and configuration
files, aimed at easing the everyday life of the user interested in performing
large parameter scans with \SIXTRACK{}.
% It describes a new set
% of UNIX BASH or Korn shell scripts which allow the use of the Berkeley Open
% Infrastructure for Network Computing, BOINC~\cite{Boinc}) as an alternative to
% the Linux LSF batch system. This note is also published and regularly updated
% on the web page
% \myhref{http://cern.ch/sixtrack-ng/doc/sixdesk/sixdesk_env.html}{cern.ch/sixtrack-ng/doc/sixdesk/sixdesk\_env.html}.
\vfill
\begin{center}
    Geneva, Switzerland \\
    \today
\end{center}

\end{titlepage}
