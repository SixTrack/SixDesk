\begin{titlepage}
\begin{center}\normalsize\scshape
    European Organization for Nuclear Research \\
    CERN BE/ABP
\end{center}
\vspace*{2mm}
\begin{flushright}
    CERN/xx/xx \\
    Updated April 2018
\end{flushright}
\begin{center}\Huge
    \textbf{SixDesk} \\
    \LARGE Version 1.0 \\
    \vspace*{8mm} the Simulation Environment for SixTrack\\
    \vspace*{8mm}\textbf{User's Reference Manual}
\end{center}
\begin{center}
  R.~De Maria, M.~Giovannozzi, E.~McIntosh, A.~Mereghetti, F.~Schmidt,
  I.~Zacharov \\
    \vspace*{4mm}Updated by:
    P.~D.~Hermes, D.~Pellegrini, S.~Kostoglou
\end{center}
\begin{center}\large
    \vspace*{10mm}\textbf{Abstract} \\
\end{center}
SixTrack is a single particle tracking code widely used at CERN. One of the
most important applications is the estimation of the dynamic aperture available
in large storage rings like the Large Hadron Collider (LHC) or the Future
Circular Collider (FCC). These studies require massive computing resources,
consisting of scans over large parameter spaces probing non-linear beam
dynamics over long times. The SixDesk environment is the simulation framework
used to manage and control the large amount of information necessary for and
produced by the studies.
\vfill
\begin{center}
    Geneva, Switzerland \\
    \today
\end{center}

\end{titlepage}
