\chapter{New Features} \label{NewFeatures}
This chapter illustrates the new features implemented in \SIXDESK{} from
the user point of view. In general, all the new features have an introduction,
where the rationale and the working principles of the new feature are briefly
presented; afterwards, an essential look at user input and implementation
is given; each section is then closed by a step by step guide, with
examples. In the following, the environment variable
\texttt{SixDeskTools} is assumed and defined as
\begin{lstlisting}
  SixDeskTools=/afs/cern.ch/project/sixtrack/SixDesk_utilities/dev
\end{lstlisting}

\section{Initialisation of Workspace and Study} \label{Initialisation}
\begin{flushright}
\emph{Original work by: A.~Mereghetti}
\end{flushright}
It is useful to have a standard way of setting up workspace and study
from within the \SIXDESK{} script, so that the user does not have to
worry about proper template files and their synchronisation with a given
version of the scripts.

\subsection{Step-by-Step Guide}
The main steps to properly set up the workspace and a study are:
\begin{enumerate}
\item set up the workspace, e.g.
\begin{lstlisting}
> $SixDeskTools/utilities/bash/set_env.sh -N scratch2/wMySpace
\end{lstlisting}
This action will set up the workspace, taking care of generating
the correct hierarchy between the \texttt{sixjobs} and the \texttt{scratch*}
directories. The action will create also the following tree structure:
\begin{lstlisting}
> cd wMySpace/sixjobs
> tree -h
.
|__ [4.0K]  control_files
|   |__ [1013]  fort.3.mother1_col
|   |__ [ 942]  fort.3.mother1_inj
|   |__ [2.0K]  fort.3.mother2_col
|   |__ [2.0K]  fort.3.mother2_col_b2
|   |__ [2.0K]  fort.3.mother2_inj
|   |__ [2.0K]  fort.3.mother2_inj_b2
|__ [ 475]  fort.3.local
|__ [4.0K]  mask
|   |__ [ 39K]  hl10BaseB1.mask
|   |__ [ 35K]  hl13B1.mask
|__ [ 996]  scan_definitions
|__ [8.2K]  sixdeskenv
|__ [ 115]  sixdesklock
|__ [4.0K]  sixdeskTaskIds
|__ [4.0K]  studies
|   |__ [   0]  sixdesklock
|__ [4.3K]  sysenv

4 directories, 14 files
\end{lstlisting}
As it can be noted, this action takes care also of making available
to the user \emph{all} template input files (see following item
for details);
\item (optional) go into the \texttt{sixjobs} dir and download templates,
  e.g.
\begin{lstlisting}
> cd wMySpace/sixjobs
> $SixDeskTools/utilities/bash/set_env.sh -n -l -c
\end{lstlisting}
This action will make available to the user the template
input files, i.e.~the \texttt{sixdeskenv}, \texttt{sysenv},
\texttt{fort.3.local} (see Sec.~\ref{fort3local}) and
\texttt{scan\_definitions} (see Sec.~\ref{ExternalScans}) files.
This action will also update
the \texttt{workspace}, \texttt{basedir} and \texttt{scratchdir}
variables in the \texttt{sixdeskenv} file
with the correct values for the workspace just set up.
Please be aware that this operation will overwrite any
pre-existing file in the \texttt{sixjobs} dir. The templates
will be downloaded from
\begin{lstlisting}
${SixDeskTools}/utilities/templates/input
\end{lstlisting}
in this way, templates and scripts are synchronised.
The \texttt{-l} option triggers the download of the
\texttt{fort.3.local} (see Sec.~\ref{fort3local}) file, whereas
the \texttt{-c} option triggers the download of the
\texttt{scan\_definitions} (see Sec.~\ref{ExternalScans}) file.
This action is optional, as it is already performed by the
\texttt{-N} action; nevertheless, it can be performed on its
own and its usage has been shown.
\end{enumerate}

When using either of the \texttt{-n} or \texttt{-N} actions,
if the user requests the \texttt{-g} option, then all the files
and directories will be downloaded with \texttt{git}. Hence, the
user can profit from the diffing tools available with \texttt{git};
on the other hand, the disk usage grows (currently $\sim$50~MB
globally after either actions).

\section{\texttt{fort.3.local}} \label{fort3local}
\begin{flushright}
\emph{Original work by: A.~Mereghetti}
\end{flushright}

\section{Enforcing the Crossing Angle} \label{EnforceXingAngle}
\begin{flushright}
\emph{Original work by: D.~Pellegrini} \\
\emph{Updated by: A.~Mereghetti}
\end{flushright}

\section{Variable Number of Angles with Amplitude} \label{varAnglesWithAmpli}
\begin{flushright}
\emph{Original work by: D.~Pellegrini} \\
\emph{Updated by: S.~Kostoglou, A.~Mereghetti}
\end{flushright}
Angular steps are forced to be consecutive, i.e.~\texttt{kstep} is
ignored and \texttt{1} is hardcoded in the logics.

\section{External Scans} \label{ExternalScans}
\begin{flushright}
\emph{Original work by: P.~D.~Hermes, D.~Pellegrini} \\
\emph{Updated by: A.~Mereghetti}
\end{flushright}
``Internal scans'' are the fundamental scans used to estimate the dynamic
aperture for a given machine configuration, mainly probing the beam phase
space via a linear scan in particle amplitude parametric in angle.
The internal scan also cover different error
configurations of the magnetic fields; optionally, the user can also request
to replicate the study varying the machine tune.
The internal scans are handled by \SIXDESK{} with the input coded
in the \texttt{sixdeskenv} file.
Table~\ref{tab:InternalScanParamters} summarises essential technical
characteristics of the internal scans.
\begin{table}[t]
\begin{center}
    \caption{Essential technical
      characteristics of the internal scans.}
    \label{tab:InternalScanParamters}
    \begin{tabular}{|c|l|l|}
    \hline
    \rowcolor{blue!30}
    \textbf{Category} & \textbf{Variable} & \textbf{Comment} \\
    \hline
    \multirowcell{2}{beam \\ phase space}
    & amplitude & main loop in \SIXDESK{}, sub-loop in \SIXTRACK{} \\
    \cline{2-3}
    & angle     & loop in \SIXDESK{}, set point in \SIXTRACK{} \\
    \hline
    \multirowcell{2}{machine \\ phase space}
    & magnetic errors (seed) & loop in \SIXDESK{}, a \MADX{} job each\\
    \cline{2-3}
    & tune & loop in \SIXDESK{}, each \SIXTRACK{} job matches the tune \\
    \hline
    \end{tabular}
\end{center}
\end{table}

A \SIXDESK{} study is exactly made of a complete internal scan, with all the
\SIXTRACK{} input files describing the machine (see Sec.~\ref{Overview})
generated by a single \texttt{*.mask} file. The beam phase space is scanned
based on the settings in \texttt{sixdeskenv} file, and machine parameters like
the multipolar errors and the tune are treated as ``close'' variations of the
original study case.

``External'' scans identify a set of additional scan parameters, not aimed at
exploring further the beam phase space but machine configurations of possible
interest -- something that could be loosely called machine ``phase space''.
Any point in
an external scan is an independent \SIXDESK{} study, and it can be handled
with the standard tools, since it has its own folders and files.
On the other hand, all the studies have something in common; hence,
it can be suitable to have a set of tools for treating all the
studies in an external scan the same way.

External scans can be useful to explore the dependence of the dynamic
aperture on parameters like chromaticity, octupole current, and crossing
angles, for the same optics. Therefore,
these scans are based on a 1:1 relation between \MADX{} and
\SIXTRACK{}, i.e.~the knobs defined in \MADX{} are exported as they
are in \SIXTRACK{} by means of the geometry files (see Sec.~\ref{Overview}).
Hence, the user is responsible for assuring that the desired parameters can be
represented by \MADX{} and all the necessary settings are propagated
to \SIXTRACK{} via the geometry input files,
including magnet kicks as computed by the \MADX{} matching.
It should be noted that no parameter defining the internal scan
coded in the \texttt{sixdeskenv} input file is modified.

Two types of external scans are available to the user:
\begin{enumerate}
\item a scan over a \emph{Cartesian grid} of an arbitrary number
  of variables with given steps for each variable. All the studies
  will be created and named after a reference machine configuration;
  each study will inherit a unique set of values of the scanned variables,
  which will appear explicitly in the study name
  together with the values actually used;
\item a scan over a \emph{preset list of studies} which must exist.
  This option is extremely useful when punctual operations are
  required on a sub-set of studies composing the original scan.
\end{enumerate}

\subsection{Input Files}
\begin{table}[t]
\begin{center}
  \caption{Parameters controlling external scans, to be defined by
    the user in the \texttt{scan\_definitions} file. The central
    block of variables is used for scans on a \emph{Cartesian grid},
    whereas the last block is used for scans on a \emph{preset list}
    of studies.}
    \label{tab:ExternalScanParameters}
    \begin{tabular}{|l|l|l|}
    \hline
    \rowcolor{blue!30}
    \textbf{Parameter Name} & \textbf{Comment} & \textbf{Example} \\
    \hline
    \texttt{scan\_masks} & trigger to use preset list of studies &
       \texttt{scan\_masks=false} \\
    \hline
    \texttt{scan\_variables} & variable names (used in study name) &
       \texttt{scan\_variables="B QP"} \\
    \texttt{scan\_vals\_<vNam>} & values to be explored for variable \texttt{<vNam>} &
    \texttt{scan\_vals\_B="1 4"} \\
    & & \texttt{scan\_vals\_QP="0 2 4"} \\
    \texttt{scan\_placeholders} & placeholders in \texttt{*.mask} file &
       \texttt{scan\_placeholders="\%BV  \%QPV"} \\
    \texttt{scan\_prefix} & common part of study name &
       \texttt{scan\_prefix="HLLHC\_inj"} \\
    \hline
    \texttt{scan\_studies} & explicit list of studies in the scan &
    \texttt{scan\_studies="HLLHC\_inj\_B\_1\_QP\_4  } \\
    & & \texttt{HLLHC\_inj\_B\_4\_QP\_0"} \\
    \hline
    \end{tabular}
\end{center}
\end{table}
The file describing the external scan is the \texttt{scan\_definitions}.
It is a new file to \SIXDESK{}, where the user fully describes the Cartesian
grid of interest or the pre-set list of studies. As for the
\texttt{sixdeskenv} and \texttt{sysenv} files, it must be coded
following the syntax of \texttt{bash}.
Table~\ref{tab:ExternalScanParameters}
lists the variables that the file should contain.
With the \texttt{scan\_masks} logical variable, the user instructs
\SIXDESK{} about the type of external scan to be performed:
\begin{description}
\item[\texttt{scan\_masks=false}] the scan is performed on the
  \emph{Cartesian grid}; in this type of scan, the central
  block of variables shown in Tab.~\ref{tab:ExternalScanParameters}
  are used;
\item[\texttt{scan\_masks=true}] the scan is performed on the
  \emph{pre-set list} of studies; in this type of scan, the last
  block of variables shown in Tab.~\ref{tab:ExternalScanParameters}
  are used.
\end{description}

It should be kept in mind that, in the case of the \emph{Cartesian grid},
the user must set up a \texttt{*.mask} file, to be used as template for
the studies in the scan. All the other regular input files
(see Sec.~\ref{Overview}) determine the
internal scan performed in each study, and are essentially cloned,
so that the dynamic aperture is probed in the same way
in all points of the external scan. On the contrary,
in the case of the preset list of studies, all the concerned
studies must be already existing, and no other input file is
required.

\subsubsection{Scan on a Cartesian Grid}
In the scan on a \emph{Cartesian grid}, all the concerned studies
are generated out of a set of template files, based on a
\texttt{sixdeskenv}, \texttt{sysenv}, \texttt{*.mask} and
\texttt{scan\_definitions} files (and \texttt{fort.3.local}, optionally).
All the optics configurations are variations of the same one coded
in the template \texttt{*.mask} file.

The user defines the parameter space in the \texttt{scan\_definitions}
file at their will, with no restrictions due to interfaces. The user must
make sure that the desired parameters can be represented by \MADX{}
and all the necessary settings are propagated to \SIXTRACK{} via
the geometry input files (see Sec.~\ref{Overview}).
In fact, contrary to what done normally in
\SIXDESK{}, the user defines suitable \emph{placeholders} that will be
used by \SIXDESK{} for query/replace in the \texttt{*.mask} file and
for disentangling the various studies. Hence, it is responsibility of the
user not only to define the variables and the concerned range of values,
but also to set up the necessary \emph{placeholders}
in the template \texttt{*.mask} file.

For starting an external scan, the user should prepare:
\begin{itemize}
\item a regular \texttt{sixdeskenv}, to be used as template.
  The file is automatically replicated by \SIXDESK{} in all the studies
  involved in the scan as is, with the exception of the actual study name
  (i.e.~the \texttt{LHCDescrip} field), which is automatically updated
  at the generation of the study.
  Hence, it is user's convenience to freeze the paramters for the internal
  scan before starting the external one, such that all the studies will inherit
  immediately the correct parameters and range of values;
\item a regular \texttt{sysenv}, to be cloned as is, with
  no further modifications by \SIXDESK{}. As for the \texttt{sixdeskenv} file, 
  it is user's convenience to set this file up correctly and completely
  before starting the external scan;
\item an optional file \texttt{fort.3.local}, to be cloned
  as is, with no further modifications by \SIXDESK{}. As for the
  \texttt{sixdeskenv} and \texttt{sysenv} files, it is user's convenience
  to set this file up correctly and completely before starting the external scan;
\item a template \texttt{*.mask} file, to be used to generate all studies
  in the scan. \SIXDESK{} will take care of cloning it to the involved studies,
  automatically performing the query/replace of the placeholders
  necessary to correctly set up the study. The query/replace patterns
  (and hence the placeholders) are uniquely defined by the user,
  and no spefic syntax is hard-coded in \SIXDESK{};
\item the \texttt{scan\_definitions} files, which contains
  the full description of the scans. More than a parameter can be scanned
  at the same time, and the actual studies handled will follow the cartesian
  product of all the parameter values.
\end{itemize}
Table~\ref{tab:ExternalScanInputFile} summarises the key facts about the
input files.
\begin{table}[t]
\begin{center}
    \caption{Input files for external scans.}
    \label{tab:ExternalScanInputFile}
    \begin{tabular}{|l|l|l|}
    \hline
    \rowcolor{blue!30}
    \textbf{File} & \textbf{Comments} & \textbf{Location} \\
    \hline
    \texttt{sixdeskenv} & -- a template file for automatic query/replace
    & \texttt{sixjobs} \\
    & -- it must define correct settings for the internal scan & \\
    \texttt{sysenv} & cloned as it is & \texttt{sixjobs} \\
    \texttt{*.mask} & -- a template for automatic query/replace & \texttt{mask} \\
    & -- it must contain place holders of scanned parameters & \\
    \texttt{scan\_definitions} & unique & \texttt{sixjobs} \\
    & it describes the scans (\texttt{bash} syntax) & \\
    \hline
    \end{tabular}
\end{center}
\end{table}

The user requests \SIXDESK{} to perform a scan on the \emph{Cartesian grid}
setting the \texttt{scan\_masks} flag in the \texttt{scan\_definitions} file
to \texttt{false}. The same file (see Tab.~\ref{tab:ExternalScanParameters})
contains all the information necessary to define the scan:
\begin{itemize}
\item the variable names to be looped on are specified by the user via
  the \texttt{scan\_variables} variable;
\item the respective placeholders in the \texttt{*.mask} file are
  specified via the \texttt{scan\_placeholders};
\item the range of values to be scanned are specified via variables
  like \texttt{scan\_vals\_<vNam>}, one per scanned parameter
  \texttt{<vNam>}.
\end{itemize}

When generating the \texttt{*.mask} specific to each study,
\SIXDESK{} will automatically copy the template \texttt{*.mask}
file and query/replace the placeholders with the actual values to be used.
Hence, the parameter names must match actual \emph{placeholders} in the
template \texttt{*.mask} file, and it is the responsibility of the
user to match the \emph{placeholders} listed in the \texttt{scan\_definitions}
with those in the template \texttt{*.mask} file.

The naming convention of the study (and hence of the \texttt{*.mask} file)
combines a commond name (which can identify e.g.~the specific optics explored in
the scan) and the name of each scanned variable
with the explicit value used in each study.

Table~\ref{tab:ExternalScanParameters} reports an example of
variables in the \texttt{scan\_definitions}, coding an external scan for
studying the dynamic aperture of the HL-LHC machine at injection; the
scan is performed on both beams (variable \texttt{B}, \texttt{\%BV}
as placeholder in \texttt{*.mask}, and values 1 and 4) with three values
of chromaticity (0, 2 and 4, variable \texttt{QP} and \texttt{\%QPV} as
placeholder in). As it can be seen, names of variables and placeholders
are fully decided by the user, with no rules enforced by \SIXDESK{}.
Anyway, at set-up time, \SIXDESK{} will check that placeholders exist in
the template \texttt{*.mask} file.

The template \texttt{*.mask} file must be existing in the \texttt{mask}
directory, and it must have the name specified in the \texttt{scan\_prefix}
field in the \texttt{scan\_definitions} file. In the example,
the template \texttt{*.mask} file would be named \texttt{HLLHC\_inj.mask}.
The actual scan is made of 6 studies, named:
\begin{lstlisting}
HLLHC_inj_B_1_QP_0
HLLHC_inj_B_1_QP_2
HLLHC_inj_B_1_QP_4
HLLHC_inj_B_4_QP_0
HLLHC_inj_B_4_QP_2
HLLHC_inj_B_4_QP_4
\end{lstlisting}

\subsubsection{Scan on a Preset List of Studies}
If the user has already produced the required \texttt{*.mask} files
and want to scan over a specific (sub)set of studies, they can
specify the study names explicitly. This can be useful if they want
to run a command for only a subset of a larger set of studies of the
Cartesian scan. To use this option, the variables used to set up
the \emph{Cartesian product}, listed in the middle block of
Tab.~\ref{tab:ExternalScanParameters} are not suitable, and
those described in last block of the same table should be used.

The user requests \SIXDESK{} to perform a scan on the \emph{preset list}
of studies setting the \texttt{scan\_masks} flag in the
\texttt{scan\_definitions} file to \texttt{true}. The same file
(see Tab.~\ref{tab:ExternalScanParameters}) specifies also the list
of the studies to be treated via their full name.
As already mentioned, the concerned studies with all their input files
and folders must already exist.

In the above example, the only studies which will be considered in the
scan are (once the \texttt{scan\_masks} flag is set to \texttt{true}
by the user):
\begin{lstlisting}
HLLHC_inj_B_1_QP_4
HLLHC_inj_B_4_QP_0
\end{lstlisting}

\subsection{Implementation}
The scans are handled via the \texttt{scans.sh} user script;
it is simply a bash wrapper which loops the action requested by the
user over the desired studies. The actual
functions are coded in \texttt{dot\_scan} (\texttt{bash}) library.
Hence, the user will have to deal with only the \texttt{scans.sh}
script.

To perform a desired action on all the studies in the scan, the user
just need to issue the \texttt{scans.sh} script using the \texttt{-x}
\emph{action} with the detailed command to be performed enclosed within
double quotes. There is no
need to specify the \texttt{-d} \emph{option} for the called script, since
\texttt{scans.sh} will automatically trigger the requeted command on each
study in the scan separately. The script
will take care of looping over all the studies and issue the requested
command on each study. The only exceptional actions that have
dedicated terminal line arguments are the generation
of the actual \texttt{*.mask} files, achieved via the \texttt{-m}
\emph{action}, and the set up of the directories of each study,
achieved via the \texttt{-s} \emph{action}.

When generating the \texttt{*.mask} files,
the script checks beforehand that all the placeholders that the
user is going to use are found in the \texttt{*.mask} template
file. To disable this option, please use the \texttt{-m} \emph{option}.

The use of the \texttt{fort.3.local} file can be triggered via
the \texttt{-l} \emph{option}, with no need to replicate it also in
the string passed through the \texttt{-x} \emph{action}.

A very basic parallelisation of the scan is available. The user can
split the final scan into smaller ones. Each of them must have
its own \texttt{scan\_definitions} files, with a unique name. Then,
the respective number of instances of the \texttt{scans.sh} can be issued,
each with the \texttt{-d} \emph{option} with the specific name
of the \texttt{scan\_definitions} instance to be used.

\subsection{Step-by-Step Guide}
This guide is given for an external scan on a \emph{Cartesian grid}
started from scratch:
\begin{enumerate}
\item set up your workspace and download template files
  (see Sec.~\ref{Initialisation});
\item edit all the necessary files, e.g.
  \begin{enumerate}
  \item \texttt{sixdeskenv} and \texttt{sysenv}, properly setting up
    the internal scans, versions of codes, etc\ldots~Please, make sure
    that the \texttt{xing} variable in \texttt{sixdeskenv} is not
    active (see Sec.~\ref{EnforceXingAngle});
  \item template \texttt{*.mask} file in the \texttt{mask} directory,
    and \texttt{scan\_definitions}. Please make sure that:
    \begin{itemize}
    \item \texttt{scan\_prefix} matches the name of the template
      \texttt{*.mask} file;
    \item the lists contained in \texttt{scan\_variables} and
      \texttt{scan\_placeholders} match;
    \item for every variable scanned (e.g.~\texttt{QP}), you have the
      corresponding list of values defined in the \texttt{scan\_vals\_*}
      (e.g.~\texttt{scan\_vals\_QP});
    \item all the placeholders defined in \texttt{scan\_placeholders}
      are actually in the \texttt{*.mask} template file, and in the
      correct positions. Please keep in mind that the query/replace
      will be performed via a \texttt{sed} command;
    \end{itemize}
  \end{enumerate}
\item generate all the necessary \texttt{*.mask} file and the
  studies, e.g.
\begin{lstlisting}
> $SixDeskTools/utilities/bash/scans.sh -m -s -l
\end{lstlisting}
The \texttt{-l} \emph{option} is illustrated in the example to show
the command in case the \texttt{fort.3.local} file is required.
The \texttt{-m} \emph{action} (i.e.~generation of \texttt{*.mask} files)
and the \texttt{-s} \emph{action} (i.e.~set up of studies) can also be
performed separately;
\item run \MADX{} and generate the geometry files for the \SIXTRACK{}
  jobs, e.g.
\begin{lstlisting}
> $SixDeskTools/utilities/bash/scans.sh -x "mad6t.sh -s"
\end{lstlisting}
Once the jobs are over, it is good practice to check them before
running \SIXTRACK{}, to avoid mis-submissions in case something
went wrong with the \MADX{} jobs. Checking can be performed e.g.
\begin{lstlisting}
> $SixDeskTools/utilities/bash/scans.sh -x "mad6t.sh -c"
\end{lstlisting}
\item submit the actual \SIXTRACK{} jobs, e.g.
\begin{lstlisting}
> $SixDeskTools/utilities/bash/scans.sh -x "run_six.sh -a -p BOINC"
\end{lstlisting}
Submission is explicitely done to the \texttt{BOINC} platform
for all the studies. The usual list of platforms supported by
\texttt{run\_six.sh} is available;
\item download results and update the job database
\begin{lstlisting}
> $SixDeskTools/utilities/bash/scans.sh -x "run_results"
\end{lstlisting}
The same command can be issued with \texttt{run\_status};
\item \texttt{scans.sh} can be used for calling any script
  in \SIXDESK{}, e.g.
\begin{lstlisting}
> $SixDeskTools/utilities/bash/scans.sh -x "correct_cases"
\end{lstlisting}
\end{enumerate}
