\chapter{New Features} \label{NewFeatures}
This chapter illustrates the new features implemented in \SIXDESK{} from
the user point of view.

\section{External Scans} \label{ExternalScans}
\begin{flushright}
\emph{Original work by: P.~.D.~Hermes, D.~Pellegrini} \\
\emph{Updated by: A.~Mereghetti}
\end{flushright}
``Internal scans'' are the scans handled by \SIXDESK{} coded
in the \texttt{sixdeskenv} file. These are the fundamental
scans used to estimate the dynamic aperture for a given machine configuration,
mainly probing the beam phase space via a linear scan in particle amplitude
parametric in angle. The internal scan can also cover different error
configurations of the magnetic fields; optionally, the user can also request
to replicate the study varying the machine tune.
Table~\ref{tab:InternalScanParamters} summarises essential technical
characteristics of the internal scans.
\begin{table}[h]
\begin{center}
    \caption{Essential technical
      characteristics of the internal scans.}
    \label{tab:InternalScanParamters}
    \begin{tabular}{|l|l|l|}
    \hline
    \rowcolor{blue!30}
    \textbf{Category} & \textbf{Variable} & \textbf{Comment} \\
    \hline
    beam phase & amplitude & loop both in \SIXTRACK{} and \SIXDESK{} \\
    space      & angle     & loop in \SIXDESK{}, set point in \SIXTRACK{} \\
    \hline
    machine & magnetic errors (seed) &
    loop in \SIXDESK{}, involving also \MADX{} \\
    phase space & tune &
    loop in \SIXDESK{}, involving re-matching in \SIXTRACK{} \\
    \hline
    \end{tabular}
\end{center}
\end{table}

The internal scans make actually one study, as all the
\SIXTRACK{} input files describing the machine (i.e.~\texttt{fort.2},
\texttt{fort.8} and \texttt{fort.16}) are generated by a single \texttt{.mask}
file. The beam phase space is scanned, and machine parameters like the
multipolar errors and the tune are treated as ``close'' variations of the
original study case.

``External'' scans identify a set of additional scan parameters, not aimed at
exploring further the beam phase space but machine configurations of possible
interest -- hence the machine ``phase space'' is explored. Any point in
an external scan is an independent \SIXDESK{} study, and it can be handled
with the standard tools, since it has its own folder trees. But since
all the studies are variations of the same machine configuration, it is
logical to keep them boundled together in the same workspace, with a convenient
tool which loops through them.

External scans be useful to explore the dependence of the dynamic aperture
on parameters like chromaticity, octupole current, and crossing angles, 
for the same linear optics. Therefore,
these scans are based on a 1:1 relation between \MADX{}
\SIXTRACK{}, i.e.~the knobs defined in \MADX{} are exported as they
are in \SIXTRACK{} by means of the geometry files (i.e.~\texttt{fort.2},
\texttt{fort.8} and \texttt{fort.16}); this includes magnet kicks as
computed by the \MADX{} matching.

It should be noted that the present machinery set up for external scans
does not modify any other parameter in the \texttt{sixdeskenv} and
\texttt{sysenv} input files.

The user can define two types of external scans:
\begin{enumerate}
\item a scan over a \emph{Cartesian grid} of an arbitrary number
  of variables with given steps for each variable;
\item a scan over a \emph{preset list} of studies which must exist beforehand,
  including the \texttt{sysenv}, \texttt{sixdeskenv} and \texttt{.mask}
  files.
\end{enumerate}

In the case of the Cartesian grid, all the studies will be named after
a reference machine configuration, and the names of the scanned variables
will appear explicitly in the study name, together with the values actually
used for a given study.

\subsection{Input Files}
The essential information is contained in the \texttt{.mask}
file, used as template for the studies in the scan,
and in the \texttt{scan\_definitions} file,
describing the parameters and range of values
of the scan. These two files determine the set of studies building
up the external scan, whereas
all the other regular input files (i.e.~\texttt{sixdeskenv},
\texttt{sysenv} and \texttt{fort.3.local}) determine the
internal scan performed in each study, and are essentially cloned,
so that all points in the external scans are probed the same way.

The user defines the parameter space at their will, with no
restriction due to interfaces; the user must make sure that
the desired parameters can be represented by \MADX{}
and all the necessary settings are propagated to \SIXTRACK{} via
the geometry input files (i.e.~\texttt{fort.2}, \texttt{fort.8}
and \texttt{fort.16}). In particular, it is responsibility of the
user not only to define the variables and range of values,
but also to prepare in the template \texttt{.mask} file
suitable \emph{placeholders} that will be recognised by
\SIXDESK{} and used for differentiating the various studies.
Hence, contrary to what done normally in
\SIXDESK{}, the user is responsible for the proper definition
of the parameter space, i.e.~not only that the range of explored
values is sensible, but also all the handles in the \texttt{.mask}
file generate adequate input to \SIXTRACK{} jobs.

More in details:
\begin{description}
\item[\texttt{sixdeskenv}] a regular \texttt{sixdeskenv} is used as template.
  The file is automatically replicated by \SIXDESK{} in all the studies
  involved in the scan as is, with the exception of the actual study name
  (i.e.~\texttt{LHCDescrip} field), which is automatically generated as well.
  Hence, it is user's convenience to freeze the paramters for the internal
  scan before starting the external one,
  such that all the studies will inherit immediately the correct ones;
\item[\texttt{sysenv}] a regular \texttt{sysenv} is duplicated as is, with
  no further modifications by \SIXDESK{}. As for the \texttt{sixdeskenv} file, 
  it is user's convenience to set this file up before starting the
  external scan;
\item[\texttt{fort.3.local}] the optional file \texttt{fort.3.local}
  can be used in the external scan with no specific limitations; it
  will be cloned as it is by \SIXDESK{} to all bundled studies. As for the
  \texttt{sixdeskenv} and \texttt{sysenv} files, it is user's convenience
  to set this file up before starting the external scan;
\item[\texttt{.mask}] a \texttt{.mask} file is used as template to all studies
  in the scan. \SIXDESK{} will take care of cloning it to the involved studies,
  automatically performing the query-replace necessary to correctly set up
  the study. The query-replace patterns are uniquely defined by the user,
  and no spefic syntax is hard-coded in \SIXDESK{};
\item[\texttt{scan\_definitions}] this is a new file to \SIXDESK{}. It contains
  the full description of the scans, using a \texttt{bash} syntax. More than a
  parameter can be scanned at the same time; if it is the case, then the actual
  studies submitted will follow the cartesian product of all the parameter
  values.
\end{description}

Table~\ref{tab:ExternalScanInputFile} summarises the key facts about the
input files.
\begin{table}[h]
\begin{center}
    \caption{Input files for external scans.}
    \label{tab:ExternalScanInputFile}
    \begin{tabular}{|l|l|l|}
    \hline
    \rowcolor{blue!30}
    \textbf{File} & \textbf{Comments} & \textbf{Location} \\
    \hline
    \texttt{sixdeskenv} & a template file for automatic query/replace
    & \texttt{sixjobs} \\
    & must define correct settings for the internal scan & \\
    \texttt{sysenv} & cloned as it is & \texttt{sixjobs} \\
    \texttt{.mask} & a template for automatic query/replace & \texttt{mask} \\
    & must contain place holders of scanned parameters & \\
    \texttt{scan\_definitions} & unique & \texttt{sixjobs} \\
    & it describes the scans (\texttt{bash} syntax) & \\
    \hline
    \end{tabular}
\end{center}
\end{table}

The user requests \SIXDESK{} to perform a scan on the Cartesian grid
or on the preset list of studies via the \texttt{scan\_masks} flag
in the \texttt{scan\_definitions} file (see later); it must be
set to \texttt{false} in case the Cartesian grid is the desired
method of scanning, of to \texttt{true} for the preset list of studies.

\subsubsection{Scan on a Cartesian Grid}
For the Cartesian grid, the variable names to be looped on in the
\texttt{.mask} files are specified by the user, together with the
the esplored range of values. The variables are automatically replaced
by \SIXDESK{} and the new \texttt{.mask} files created. The user defines
the parameters of the Cartesian grid and the range of values that should be
covered by each. In each new study, the same template \texttt{.mask}
file is copied and modified according to the requirements of the user.
The parameter names must match actual \emph{placeholders} in the
\texttt{.mask} file. The variable names to use in the \texttt{.mask}
file and the values to be spanned are set in the \texttt{scan\_definitions}
file.

The naming convention of the study (and of the \texttt{.mask} file)
combines a commond name (which can identify the specific optics explored in
the scan, for instance) and the name of each scanned variable
with the explicit value used in each study.

\begin{table}[h]
\begin{center}
  \caption{Parameters controlling external scans on a Cartesian
    grid, to be defined by the user in the \texttt{scan\_definitions} file.}
    \label{tab:ExternalScanParametersCartesian}
    \begin{tabular}{|l|l|l|}
    \hline
    \rowcolor{blue!30}
    \textbf{Parameter Name} & \textbf{Comment} & \textbf{Example} \\
    \hline
    \texttt{scan\_variables} & variable names (used in study name) &
       \texttt{scan\_variables="B QP"} \\
    \texttt{scan\_vals\_<vNam>} & values to be explored for variable \texttt{<vNam>} &
    \texttt{scan\_vals\_B="1 4"} \\
    & & \texttt{scan\_vals\_QP="0 2 4"} \\
    \texttt{scan\_placeholders} & placeholders in \texttt{.mask} file &
       \texttt{scan\_placeholders="\%BV  \%QPV"} \\
    \texttt{scan\_prefix} & common part of study name &
       \texttt{scan\_prefix="HLLHC\_inj"} \\
    \texttt{scan\_masks} & trigger to use preset list of studies &
       \texttt{scan\_masks=false} \\
    \hline
    \end{tabular}
\end{center}
\end{table}
The user defines in the \texttt{scan\_definitions} files the variable names,
the placeholders used in the \texttt{.mask} file and the actual values to be
scanned. Table~\ref{tab:ExternalScanParametersCartesian} summarises the necessary
fields, with examples.

The examples report the parameters scans for
studying the dynamic aperture of the HL-LHC machine at injection; the
scans are done on both beams (1 and 4, variable \texttt{B} and \texttt{\%BV}
as placeholder in \texttt{.mask}) with three values of chromaticity
(0, 2 and 4, variable \texttt{QP} and \texttt{\%QPV} as placeholder in
\texttt{.mask}). Names of variables and placeholders are fully decided by
the user, with no rules enforced by \SIXDESK{}. Anyway, at set-up time,
\SIXDESK{} will check that placeholders exist in the template
\texttt{.mask} file.

The template \texttt{.mask} file must be existing in the \texttt{mask}
directory, and it must have the name specified in the \texttt{scan\_prefix}
field in the \texttt{scan\_definitions} file. In the above example,
the template \texttt{.mask} file would be named \texttt{HLLHC\_inj.mask}.
The actual scan is made of 6 studies, named:
\begin{verbatim}
HLLHC_inj_B_1_QP_0
HLLHC_inj_B_1_QP_2
HLLHC_inj_B_1_QP_4
HLLHC_inj_B_4_QP_0
HLLHC_inj_B_4_QP_2
HLLHC_inj_B_4_QP_4
\end{verbatim}

\subsubsection{Scan on a Preset List}
If the user has already produced the required \texttt{.mask} files
and wants to scan over a specific (sub)set of studies, he can also
specify the study names explicitly. This can be useful if we want
to run a command for only a subset of a larger set of studies of the
Cartesian scan. To use this option, the variable listed in
Tab.~\ref{tab:ExternalScanParametersCartesian} are not suitable, and
those described in Tab.~\ref{ExternalScanParametersPresetList} should be used.
\begin{table}[h]
\begin{center}
  \caption{Parameters controlling external scans for a preset list
    of studie, to be defined by the user in the \texttt{scan\_definitions}
    file.}
    \label{tab:ExternalScanParametersPresetList}
    \begin{tabular}{|l|l|l|}
    \hline
    \rowcolor{blue!30}
    \textbf{Parameter Name} & \textbf{Comment} & \textbf{Example} \\
    \hline
    \texttt{scan\_masks} & trigger to use preset list of studies &
       \texttt{scan\_masks=true} \\
    \texttt{scan\_studies} & explicit list of studies in the scan &
    \texttt{scan\_studies="HLLHC\_inj\_B\_1\_QP\_4  } \\
     & & \texttt{HLLHC\_inj\_B\_4\_QP\_0"} \\
    \hline
    \end{tabular}
\end{center}
\end{table}
In the above example, the only studies which will be considered in the
scan are:
\begin{verbatim}
HLLHC_inj_B_1_QP_4
HLLHC_inj_B_4_QP_0
\end{verbatim}
As already mentioned, all the necessary input files (e.g.~\texttt{.mask},
\texttt{sixdeskenv}, \texttt{sysenv}, \texttt{fort.3.local}, etc\ldots)
and the concerned folders (e.g.~\texttt{sixtrack\_input}, \texttt{track},
\texttt{work}, \texttt{study}, etc\ldots) must already exist.

\subsection{Step-by-Step Guide}
This little guide will show the user how to set-up an the external
scan and how to proceed, step by step.

\subsection{Implementation}
Everything is contained in the \texttt{scans.sh} user script
and the \texttt{dot\_scan} \texttt{bash} library.
