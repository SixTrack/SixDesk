\chapter{New Features} \label{NewFeatures}
This chapter illustrates the new features implemented in \SIXDESK{} from
the user point of view.

\section{Initialisation of Workspace and Study} \label{Initialisation}
\begin{flushright}
\emph{Original work by: A.~Mereghetti}
\end{flushright}
It is useful to have a standard way of setting up workspace and study
from within the \SIXDESK{} script, so that the user does not have to
worry about proper template files, to be kept in sync with a given
version of the scripts.

\subsection{Step-by-Step Guide}
This short guide will show the user how to properly initialise
a new workspace and study, step by step. In the following,
the environment variable \texttt{SixDeskTools} is defined, e.g.~as
\begin{lstlisting}
  SixDeskTools=/afs/cern.ch/project/sixtrack/SixDesk_utilities/dev
\end{lstlisting}
\begin{enumerate}
\item set up the workspace, e.g.
\begin{lstlisting}
> $SixDeskTools/utilities/bash/set_env.sh -N scratch2/wMySpace
\end{lstlisting}
This action will set up the workspace, taking care of generating
the correct hierarchy between the \texttt{sixjobs} and the \texttt{scratch*}
directories. The action will create also the following tree structure:
\begin{lstlisting}
> cd wMySpace/sixjobs
> tree -h
.
|__ [2.0K]  control_files
|   |__ [1013]  fort.3.mother1_col
|   |__ [ 942]  fort.3.mother1_inj
|   |__ [2.0K]  fort.3.mother2_col
|   |__ [2.0K]  fort.3.mother2_col_b2
|   |__ [2.0K]  fort.3.mother2_inj
|   |__ [2.0K]  fort.3.mother2_inj_b2
|__ [2.0K]  mask
|   |__ [ 39K]  hl10BaseB1.mask
|   |__ [ 35K]  hl13B1.mask
|   [   0]  sixdesklock
|__ [2.0K]  sixdeskTaskIds
|__ [2.0K]  studies
    |__ [   0]  sixdesklock

4 directories, 10 files
\end{lstlisting}
\item go into the \texttt{sixjobs} dir and download templates, e.g.
\begin{lstlisting}
> cd wMySpace/sixjobs
> $SixDeskTools/utilities/bash/set_env.sh -n -l -c
\end{lstlisting}
This action will generate the \texttt{sixdeskenv}, \texttt{sysenv},
\texttt{fort.3.local} and \texttt{scan\_definitions} (see
Sec.~\ref{Initialisation}) files. This action will also update
the \texttt{workspace}, \texttt{basedir} and \texttt{scratchdir}
variables in the \texttt{sixdeskenv} file
with the correct values for the workspace just set up.
Please be aware that this operation will overwrite any
pre-existing file in the \texttt{sixjobs} dir. The templates
will be downloaded from
\texttt{\${SixDeskTools}/utilities/templates/input};
in this way, templates and scripts are synchronised.
\end{enumerate}

\section{\texttt{fort.3.local}} \label{fort3local}
\begin{flushright}
\emph{Original work by: A.~Mereghetti}
\end{flushright}

\section{Enforcing the Crossing Angle} \label{EnforceXingAngle}
\begin{flushright}
\emph{Original work by: D.~Pellegrini} \\
\emph{Updated by: A.~Mereghetti}
\end{flushright}

\section{Variable Number of Angles with Amplitude} \label{varAnglesWithAmpli}
\begin{flushright}
\emph{Original work by: D.~Pellegrini} \\
\emph{Updated by: S.~Kostoglou, A.~Mereghetti}
\end{flushright}

\section{External Scans} \label{ExternalScans}
\begin{flushright}
\emph{Original work by: P.~D.~Hermes, D.~Pellegrini} \\
\emph{Updated by: A.~Mereghetti}
\end{flushright}
``Internal scans'' are the scans handled by \SIXDESK{} coded
in the \texttt{sixdeskenv} file. These are the fundamental
scans used to estimate the dynamic aperture for a given machine configuration,
mainly probing the beam phase space via a linear scan in particle amplitude
parametric in angle. The internal scan also cover different error
configurations of the magnetic fields; optionally, the user can also request
to replicate the study varying the machine tune.
Table~\ref{tab:InternalScanParamters} summarises essential technical
characteristics of the internal scans.
\begin{table}[h]
\begin{center}
    \caption{Essential technical
      characteristics of the internal scans.}
    \label{tab:InternalScanParamters}
    \begin{tabular}{|c|l|l|}
    \hline
    \rowcolor{blue!30}
    \textbf{Category} & \textbf{Variable} & \textbf{Comment} \\
    \hline
    \multirowcell{2}{beam \\ phase space}
    & amplitude & main loop in \SIXDESK{}, sub-loop in \SIXTRACK{} \\
    \cline{2-3}
    & angle     & loop in \SIXDESK{}, set point in \SIXTRACK{} \\
    \hline
    \multirowcell{2}{machine \\ phase space}
    & magnetic errors (seed) & loop in \SIXDESK{}, a \MADX{} job each\\
    \cline{2-3}
    & tune & loop in \SIXDESK{}, each \SIXTRACK{} job matches the tune \\
    \hline
    \end{tabular}
\end{center}
\end{table}

A \SIXDESK{} study is exactly made of a complete internal scan, with all the
\SIXTRACK{} input files describing the machine (see Sec.~\ref{Overview})
generated by a single \texttt{*.mask} file. The beam phase space is scanned
based on the settings in \texttt{sixdeskenv} file, and machine parameters like
the multipolar errors and the tune are treated as ``close'' variations of the
original study case.

``External'' scans identify a set of additional scan parameters, not aimed at
exploring further the beam phase space but machine configurations of possible
interest -- something we may call machine ``phase space''. Any point in
an external scan is an independent \SIXDESK{} study, and it can be handled
with the standard tools, since it has its own folders and files.

External scans can be useful to explore the dependence of the dynamic
aperture on parameters like chromaticity, octupole current, and crossing
angles, for the same optics. Therefore,
these scans are based on a 1:1 relation between \MADX{} and
\SIXTRACK{}, i.e.~the knobs defined in \MADX{} are exported as they
are in \SIXTRACK{} by means of the geometry files (see Sec.~\ref{Overview}).
Hence, the user must make sure that the desired parameters can be
represented by \MADX{} and all the necessary settings are propagated
to \SIXTRACK{} via the geometry input files,
including magnet kicks as computed by the \MADX{} matching.

It should be noted that no parameter defining the internal scan
coded in the \texttt{sixdeskenv} input file is modified.

Two types of external scans are available to the user:
\begin{enumerate}
\item a scan over a \emph{Cartesian grid} of an arbitrary number
  of variables with given steps for each variable. All the studies
  will be named after a reference machine configuration, and the names
  of the scanned variables will appear explicitly in the study name,
  together with the values actually used for a given study.
\item a scan over a \emph{preset list} of studies which must exist.
  This option is extremely useful when punctual operations are
  required on a sub-set of studies composing the original scan.
\end{enumerate}

\subsection{Input Files}
\begin{table}[h]
\begin{center}
  \caption{Parameters controlling external scans, to be defined by
    the user in the \texttt{scan\_definitions} file.}
    \label{tab:ExternalScanParametersCartesian}
    \begin{tabular}{|l|l|l|}
    \hline
    \rowcolor{blue!30}
    \textbf{Parameter Name} & \textbf{Comment} & \textbf{Example} \\
    \hline
    \texttt{scan\_masks} & trigger to use preset list of studies &
       \texttt{scan\_masks=false} \\
    \hline
    \texttt{scan\_variables} & variable names (used in study name) &
       \texttt{scan\_variables="B QP"} \\
    \texttt{scan\_vals\_<vNam>} & values to be explored for variable \texttt{<vNam>} &
    \texttt{scan\_vals\_B="1 4"} \\
    & & \texttt{scan\_vals\_QP="0 2 4"} \\
    \texttt{scan\_placeholders} & placeholders in \texttt{*.mask} file &
       \texttt{scan\_placeholders="\%BV  \%QPV"} \\
    \texttt{scan\_prefix} & common part of study name &
       \texttt{scan\_prefix="HLLHC\_inj"} \\
    \hline
    \texttt{scan\_studies} & explicit list of studies in the scan &
    \texttt{scan\_studies="HLLHC\_inj\_B\_1\_QP\_4  } \\
    & & \texttt{HLLHC\_inj\_B\_4\_QP\_0"} \\
    \hline
    \end{tabular}
\end{center}
\end{table}
The file describing the external scan is the \texttt{scan\_definitions}.
It is a new file to \SIXDESK{}, where the user fully describes the Cartesian
grid of interest or the pre-set list of studies.
Table~\ref{tab:ExternalScanParameters}
lists the variables that the file should contain.
With the \texttt{scan\_masks} logical variable, the user instructs
\SIXDESK{} about the type of external scan to be performed:
\begin{description}
\item[\texttt{scan\_masks=false}] the scan is performed on the
  \emph{Cartesian grid}; in this type of scan, the central
  block of variables shown in Tab.~\ref{tab:ExternalScanParameters}
  are used;
\item[\texttt{scan\_masks=true}] the scan is performed on the
  \emph{pre-set list} of studies; in this type of scan, the last
  block of variables shown in Tab.~\ref{tab:ExternalScanParameters}
  are used.
\end{description}

It should be kept in mind that, in the case of the \emph{Cartesian grid},
the user must set up a \texttt{*.mask} file, to be used as template for
the studies in the scan. All the other regular input files
(see Sec.~\ref{Overview}) determine the
internal scan performed in each study, and are essentially cloned,
so that the dynamic aperture is probed in the same way
in all points of the external scan. On the contrary,
in the case of the preset list of studies, all the concerned
studies must be already existing, and no other input file is
required.

\subsubsection{Scan on a Cartesian Grid}
The user defines the parameter space in the \texttt{scan\_definitions}
at their will, with no restriction due to interfaces. The user must
make sure that the desired parameters can be represented by \MADX{}
and all the necessary settings are propagated to \SIXTRACK{} via
the geometry input files (see Sec.~\ref{Overview}).
In fact, the user 

In particular, it is responsibility of the
user not only to define the variables and range of values,
but also to prepare in the template \texttt{*.mask} file
suitable \emph{placeholders} that will be recognised by
\SIXDESK{} and used for differentiating the various studies.
Hence, contrary to what done normally in
\SIXDESK{}, the user is responsible for the proper definition
of the parameter space, i.e.~not only that the range of explored
values is sensible, but also all the handles in the \texttt{*.mask}
file generate adequate input to \SIXTRACK{} jobs.

More in details:
\begin{description}
\item[\texttt{sixdeskenv}] a regular \texttt{sixdeskenv} is used as template.
  The file is automatically replicated by \SIXDESK{} in all the studies
  involved in the scan as is, with the exception of the actual study name
  (i.e.~\texttt{LHCDescrip} field), which is automatically updated.
  Hence, it is user's convenience to freeze the paramters for the internal
  scan before starting the external one, such that all the studies will inherit
  immediately the correct parameters and range of values;
\item[\texttt{sysenv}] a regular \texttt{sysenv} is duplicated as is, with
  no further modifications by \SIXDESK{}. As for the \texttt{sixdeskenv} file, 
  it is user's convenience to set this file up before starting the
  external scan;
\item[\texttt{fort.3.local}] the optional file \texttt{fort.3.local}
  can be used in the external scan with no specific limitations; it
  will be cloned as it is by \SIXDESK{} to all bundled studies. As for the
  \texttt{sixdeskenv} and \texttt{sysenv} files, it is user's convenience
  to set this file up before starting the external scan;
\item[\texttt{*.mask}] a \texttt{*.mask} file is used as template to all studies
  in the scan. \SIXDESK{} will take care of cloning it to the involved studies,
  automatically performing the query-replace of the placeholders
  necessary to correctly set up the study. The query-replace patterns
  (and hence the placeholders) are uniquely defined by the user,
  and no spefic syntax is hard-coded in \SIXDESK{};
\item[\texttt{scan\_definitions}] it contains
  the full description of the scans, using a \texttt{bash} syntax. More than a
  parameter can be scanned at the same time; if it is the case, then the actual
  studies submitted will follow the cartesian product of all the parameter
  values.
\end{description}

Table~\ref{tab:ExternalScanInputFile} summarises the key facts about the
input files.
\begin{table}[h]
\begin{center}
    \caption{Input files for external scans.}
    \label{tab:ExternalScanInputFile}
    \begin{tabular}{|l|l|l|}
    \hline
    \rowcolor{blue!30}
    \textbf{File} & \textbf{Comments} & \textbf{Location} \\
    \hline
    \texttt{sixdeskenv} & a template file for automatic query/replace
    & \texttt{sixjobs} \\
    & must define correct settings for the internal scan & \\
    \texttt{sysenv} & cloned as it is & \texttt{sixjobs} \\
    \texttt{*.mask} & a template for automatic query/replace & \texttt{mask} \\
    & must contain place holders of scanned parameters & \\
    \texttt{scan\_definitions} & unique & \texttt{sixjobs} \\
    & it describes the scans (\texttt{bash} syntax) & \\
    \hline
    \end{tabular}
\end{center}
\end{table}

The user requests \SIXDESK{} to perform a scan on the \emph{Cartesian grid}
setting the \texttt{scan\_masks} flag in the \texttt{scan\_definitions} file
to \texttt{false}. The same file (see
Tab.~\ref{tab:ExternalScanParametersCartesian}) contains all the information
necessary to perform the scan:
\begin{itemize}
\item the variable names to be looped on are specified by the user via
  the \texttt{scan\_variables} variable;
\item the respective placeholders in the \texttt{*.mask} file are
  specified via the \texttt{scan\_placeholders};
\item the range of values to be scanned are specified via variables
  like \texttt{scan\_vals\_<vNam>}, one per scanned parameter
  \texttt{<vNam>}.
\end{itemize}

When generating the \texttt{*.mask} specific to each study,
\SIXDESK{} will automatically query-replace the placeholders with
the actual values to be used.

The variables are automatically
replaced by \SIXDESK{} and the new \texttt{*.mask} files created. The user defines
the parameters of the Cartesian grid and the range of values that should be
covered by each. In each new study, the same template \texttt{*.mask}
file is copied and modified according to the requirements of the user.
The parameter names must match actual \emph{placeholders} in the
\texttt{*.mask} file. The variable names to use in the \texttt{*.mask}
file and the values to be spanned are set in the \texttt{scan\_definitions}
file.

The naming convention of the study (and of the \texttt{*.mask} file)
combines a commond name (which can identify the specific optics explored in
the scan, for instance) and the name of each scanned variable
with the explicit value used in each study.

The user defines in the \texttt{scan\_definitions} files the variable names,
the placeholders used in the \texttt{*.mask} file and the actual values to be
scanned. Table~\ref{tab:ExternalScanParametersCartesian} summarises the necessary
fields, with examples.

The examples report the parameters scans for
studying the dynamic aperture of the HL-LHC machine at injection; the
scans are done on both beams (1 and 4, variable \texttt{B} and \texttt{\%BV}
as placeholder in \texttt{*.mask}) with three values of chromaticity
(0, 2 and 4, variable \texttt{QP} and \texttt{\%QPV} as placeholder in
\texttt{*.mask}). Names of variables and placeholders are fully decided by
the user, with no rules enforced by \SIXDESK{}. Anyway, at set-up time,
\SIXDESK{} will check that placeholders exist in the template
\texttt{*.mask} file.

The template \texttt{*.mask} file must be existing in the \texttt{mask}
directory, and it must have the name specified in the \texttt{scan\_prefix}
field in the \texttt{scan\_definitions} file. In the above example,
the template \texttt{*.mask} file would be named \texttt{HLLHC\_inj.mask}.
The actual scan is made of 6 studies, named:
\begin{lstlisting}
HLLHC_inj_B_1_QP_0
HLLHC_inj_B_1_QP_2
HLLHC_inj_B_1_QP_4
HLLHC_inj_B_4_QP_0
HLLHC_inj_B_4_QP_2
HLLHC_inj_B_4_QP_4
\end{lstlisting}

\subsubsection{Scan on a Preset List}
If the user has already produced the required \texttt{*.mask} files
and wants to scan over a specific (sub)set of studies, he can also
specify the study names explicitly. This can be useful if we want
to run a command for only a subset of a larger set of studies of the
Cartesian scan. To use this option, the variable listed in
Tab.~\ref{tab:ExternalScanParametersCartesian} are not suitable, and
those described in Tab.~\ref{tab:ExternalScanParametersPresetList} should be used.
\begin{table}[h]
\begin{center}
  \caption{Parameters controlling external scans for a preset list
    of studie, to be defined by the user in the \texttt{scan\_definitions}
    file.}
    \label{tab:ExternalScanParametersPresetList}
    \begin{tabular}{|l|l|l|}
    \hline
    \rowcolor{blue!30}
    \textbf{Parameter Name} & \textbf{Comment} & \textbf{Example} \\
    \hline
    \end{tabular}
\end{center}
\end{table}
In the above example, the only studies which will be considered in the
scan are:
\begin{lstlisting}
HLLHC_inj_B_1_QP_4
HLLHC_inj_B_4_QP_0
\end{lstlisting}
As already mentioned, all the necessary input files (e.g.~\texttt{*.mask},
\texttt{sixdeskenv}, \texttt{sysenv}, \texttt{fort.3.local}, etc\ldots)
and the concerned folders (e.g.~\texttt{sixtrack\_input}, \texttt{track},
\texttt{work}, \texttt{study}, etc\ldots) must already exist.

\subsection{Implementation}
The scans are handled via the \texttt{scans.sh} user script;
it is a bash wrapper, looping the action requested by the
user over the desired studies. The actual
functions are coded in \texttt{dot\_scan} (\texttt{bash}) library.
Hence, the user will have to deal with only the \texttt{scans.sh}
script.

To perform a desired action on all the studies in the scan, the user
just need to issue the \texttt{scans.sh} script using the \texttt{-x}
\emph{action} with the detailed command within double quotes. There is no
need to specify the \texttt{-d} \emph{option} of the called script, since
\texttt{scans.sh} will automatically trigger the requeted command on each
study in the scan separately. The script
will take care of looping over all the studies and issue the requested
command on each study. The only exceptions are the generation
of the actual \texttt{*.mask} files, achieved via the \texttt{-m}
\emph{action}, and the set up of the directories of each study,
achieved via the \texttt{-s} \emph{action}.

When generating the \texttt{*.mask} files,
the script checks beforehand that all the placeholders that the
user is going to use are found in the \texttt{*.mask} template
file. To disable this option, please use the \texttt{-m} \emph{option}.

The use of the \texttt{fort.3.local} file can be triggered via
the \texttt{-l} \emph{option}, with no need to replicate it also in
the string passed through the \texttt{-x} \emph{action}.

A very basic parallelisation of the scan is available. The user can
split the final scan into smaller ones. Each of them must have
its own \texttt{scan\_definitions} files, with a unique name. Then,
the same number of instances of the \texttt{scans.sh} can be issued,
each with the \texttt{-d} \emph{option}, specifying the name
of the \texttt{scan\_definitions} instance to be used.

\subsection{Step-by-Step Guide}
This little guide will show the user how to set-up an the external
scan and how to proceed, step by step. The reference guide is given
for a scan on a Cartesian grid started from scratch, whereas some
remarks will be done for the preset list of studies.

\begin{enumerate}
\item set up your workspace and download template files
  (see Sec.~\ref{Initialisation});
\item edit all the necessary files, e.g.
  \begin{enumerate}
  \item \texttt{sixdeskenv} and \texttt{sysenv}, properly setting up
    the internal scans, versions of codes, etc\ldots~Please, make sure
    that the \texttt{xing} variable in \texttt{sixdeskenv} is not
    active (see Sec.~\ref{EnforceXingAngle});
  \item template \texttt{*.mask} file in the \texttt{mask} directory,
    and \texttt{scan\_definitions}. Please make sure that:
    \begin{itemize}
    \item \texttt{scan\_prefix} matches the name of the template
      \texttt{*.mask} file;
    \item the lists contained in \texttt{scan\_variables} and
      \texttt{scan\_placeholders} match;
    \item for every variable scanned (e.g.~\texttt{QP}), you have the
      corresponding list of values defined in the \texttt{scan\_vals\_*}
      (e.g.~\texttt{scan\_vals\_QP});
    \item all the placeholders defined in \texttt{scan\_placeholders}
      are actually in the \texttt{*.mask} template file, and in the
      correct positions. Please keep in mind that the query/replace
      will be performed via a \texttt{sed} command.
    \end{itemize}
  \end{enumerate}
\item generate all the necessary \texttt{*.mask} file and the
  studies, e.g.
\begin{lstlisting}
> $SixDeskTools/utilities/bash/scans.sh -m -s -l
\end{lstlisting}
The \texttt{-l} \emph{option} is illustrated in the example.
If the \texttt{fort.3.local} file is not required, please ignore it.
The \texttt{-m} \emph{action} (i.e.~generation of \texttt{*.mask} files)
and the \texttt{-s} \emph{action} (i.e.~set up of studies) can be
performed separately;
\item run \MADX{} and generate the geometry files for the \SIXTRACK{}
  jobs, e.g.
\begin{lstlisting}
> $SixDeskTools/utilities/bash/scans.sh -x "mad6t.sh -s"
\end{lstlisting}
Once the jobs are over, it is good practice to check them before
running \SIXTRACK{}, to avoid mis-submissions in case something
went wrong with the \MADX{} jobs. Checking can be performed e.g.
\begin{lstlisting}
> $SixDeskTools/utilities/bash/scans.sh -x "mad6t.sh -c"
\end{lstlisting}
\item submit the actual \SIXTRACK{} jobs, e.g.
\begin{lstlisting}
> $SixDeskTools/utilities/bash/scans.sh -x "run_six.sh -a -p BOINC"
\end{lstlisting}
\item download results and update job database
\begin{lstlisting}
> $SixDeskTools/utilities/bash/scans.sh -x "run_results"
\end{lstlisting}
\end{enumerate}
