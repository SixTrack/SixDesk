\chapter{New Features} \label{NewFeatures}
This chapter illustrates the new features implemented in \SIXDESK{} from
the user point of view.

\section{External Scans} \label{ExternalScans}
\begin{flushright}
\emph{Original work by: P.~.D.~Hermes, D.~Pellegrini} \\
\emph{Updated by: A.~Mereghetti}
\end{flushright}
``Internal scans'' are the scans handled by \SIXDESK{} coded
in the \texttt{sixdeskenv} file. These are the fundamental
scans used to estimate the dynamic aperture for a given machine configuration,
mainly probing the beam phase space via a linear scan in particle amplitude
parametric in angle. The internal scan can also cover different error
configurations of the magnetic fields; optionally, the user can also request
to replicate the study varying the machine tune.
Table~\ref{tab:InternalScanParamters} summarises essential technical
characteristics of the internal scans.
\begin{table}[h]
\begin{center}
    \caption{Essential technical
      characteristics of the internal scans.}
    \label{tab:InternalScanParamters}
    \begin{tabular}{|l|l|l|}
    \hline
    \rowcolor{blue!30}
    \textbf{Category} & \textbf{Variable} & \textbf{Comment} \\
    \hline
    beam phase & amplitude & loop both in \SIXTRACK{} and \SIXDESK{} \\
    space      & angle     & loop in \SIXDESK{}, set point in \SIXTRACK{} \\
    \hline
    machine & magnetic errors (seed) &
    loop in \SIXDESK{}, involving also \MADX{} \\
    phase space & tune &
    loop in \SIXDESK{}, involving re-matching in \SIXTRACK{} \\
    \hline
    \end{tabular}
\end{center}
\end{table}

The internal scans make actually one study, as all the
\SIXTRACK{} input files describing the machine (i.e.~\texttt{fort.2},
\texttt{fort.8} and \texttt{fort.16}) are generated by a single \texttt{.mask}
file. The beam phase space is scanned, and machine parameters like the
multipolar errors and the tune are treated as ``close'' variations of the
original study case.

``External'' scans identify a set of additional scan parameters, not aimed at
exploring further the beam phase space but machine configurations of possible
interest -- hence the machine ``phase space'' is explored. Any point in
an external scan is an independent \SIXDESK{} study, but all the studies
are variations of the same machine configuration. Hence, it is
logical to keep the studies boundled together in the same workspace.
All the studies will be named after a reference machine configuration,
and the names of the scanned variables will appear explicitly in the study
name, together with the values actually used for a given study.

External scans be useful to explore the dependence of the dynamic aperture
on parameters like chromaticity, octupole current, and crossing angles, 
for the same linear optics. Therefore,
these scans are based on a 1:1 relation between \MADX{}
\SIXTRACK{}, i.e.~the knobs defined in \MADX{} are exported as they
are in \SIXTRACK{}; this includes magnet kicks as computed by the
\MADX{} matching.

\subsection{Input Files}
The essential information is contained in the \texttt{.mask}
file, and in the \texttt{scan\_definitions} file,
describing the parameters and range of values
of the scan. These two files determine the set of studies building
up the external scan, whereas
all the other regulr input files (i.e.~\texttt{sixdeskenv},
\texttt{sysenv} and \texttt{fort.3.local}) determine the
internal scan performed in each study, and are essentially cloned.

The user defines the parameter space at their will, with no
restriction due to interfaces; the user must make sure that
the desired parameters can be represented by \MADX{}
and all the necessary settings are propagated to \SIXTRACK{} via
the geometry input files (i.e.~\texttt{fort.2}, \texttt{fort.8}
and \texttt{fort.16}). PLACEHOLDERS
Hence, contrary to what done normally in
\SIXDESK{}, the user is responsible for the proper definition
of the parameter space, i.e.~not only that the range of explored
values is sensible, but also all the handles in the \texttt{.mask}
file generate adequate input to \SIXTRACK{} jobs.

More in details:
\begin{description}
\item[\texttt{sixdeskenv}] a regular \texttt{sixdeskenv} is used as template.
  The file is automatically replicated in all the studies in the scan as is,
  with the exception of the actual study name, which is automatically generated
  as well. Hence, for the sake of clarity,  it is user's convenience to freeze
  the paramters for the internal scan before performing the external scan,
  such that all the studies will inherit immediately the correct ones;
\item[\texttt{sysenv}] a regular \texttt{sysenv} is duplicated as is, with
  no further modification. As for the \texttt{sixdeskenv} file, 
  it is user's convenience to set up this file before performing the
  external scan;
\item[\texttt{fort.3.local}] the optional file \texttt{fort.3.local}
  can be used in the external scan with no specific limitations, and it
  will be cloned as it is to all bundled studies. As for the
  \texttt{sixdeskenv} and \texttt{sysenv} files, it is user's convenience
  to set up this file before performing the external scan;
\item[\texttt{.mask}] a template \texttt{.mask} file of the scan is used
  for generating as template.
\end{description}

\begin{table}[h]
\begin{center}
    \caption{Input files for external scans.}
    \label{tab:ExternalScanInputFile}
    \begin{tabular}{|l|l|l|}
    \hline
    \rowcolor{blue!30}
    \textbf{File} & \textbf{Comment} & \textbf{Location} \\
    \hline
    \texttt{sixdeskenv} & a template one, with correct settings for the internal scan & \texttt{sixjobs} \\
    \texttt{sysenv} & to be replicated as is & \texttt{sixjobs} \\
    \texttt{.mask} & the template \texttt{.mask} file, containing the place holders & \texttt{sixjobs} \\
    \texttt{scan\_definitions} & description of the scans (bash) & \texttt{sixjobs} \\
    \hline
    \end{tabular}
\end{center}
\end{table}
