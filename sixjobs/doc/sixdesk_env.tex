\documentclass{cernatsnote}    % Specifies the document style.
\usepackage{times}
\usepackage{epsfig} 
\usepackage{html}

\textheight 23cm
\voffset -2cm

\begin{document}
\begin{titlepage}
\documentlabel{draft}
\keywords{software, sixtrack, dynamic aperture}
\date{\today}
\title{The SixDesk Run Environment for SixTrack} 
\author{E.~McIntosh, R.~De~Maria}
\email{Eric.McIntosh@cern.ch}

\maketitle
 
\begin{abstract}
  This document replaces the ``Run Environment for 
  SixTrack''~\cite{Runsix},
  and describes how massive tracking campaigns can be
  performed with ``SixTrack''~\cite{SixTrack} starting from a MAD-X input
  file of the LHC lattice, the so called mask file. It describes a new set 
  of UNIX BASH or Korn shell scripts which allow the use of 
  the Berkeley Open Infrastructure for
 Network Computing, BOINC~\cite{Boinc}) as an alternative to the Linux LSF batch system.
\end{abstract}

\end{titlepage}
%
Version 24 6th October, 2012.

\section{Objectives}
%
The principal objective of the SixDesk run environment was to allow a physicist to
run a tracking campaign, on either the CERN LSF batch system or BOINC,
using the familiar SixTrack run environment on Linux. 
At the same time, the opportunity was taken to group all
user modifiable parameters into two files, {\tt sixdeskenv} and {\tt sysenv}, 
and to speed up
{\tt run\_mad6t}, the madX to SixTrack conversion, by running in parallel in batch. 
The specification of tune scans, fractional amplitude ranges and steps, and various
other physics options were also added to the {\tt run\_six} script/command.
All SixDesk scripts report activity to the {\tt sixdesk.log} file
and exit with an error code if an unrecoverable error is detected.
The messages are optionally sent to {\tt STDOUT}, your screen, in an interactive
session.
%
\section{Getting Started}
First note that in order to use BOINC you must not only have a CERN account
but must also be registered in the
AFS protection group boinc:users. This is done with the script
{\tt add\_boinc\_user} but only E. McIntosh, M. Giovannozzi and
IT BOINC administrators are authorised to do so.

All the user modifiable parameters have been collected 
into the two scripts {\tt sixdeskenv, sysenv} obviating the need to look through all
scripts and make matching changes, simplifying the usage of the scripts,
and limiting the risk of error.

It is assumed that there is a large amount (at least 1GB) of disk space
available to be used as a workspace which we shall call
{\tt w1} for illustration. The {\tt w1} directory must be a link in \$HOME
to a directory with the same name, usually on an scratch disk 
(AFS scratch0, scratch1, etc or in a /scratch directory).
Note that by default access to all AFS files will be limited; it is useful
to do a {\tt fs setacl w1 system:anyuser rl} so that everything in an
AFS workspace can be read by support.

A workspace is created by e.g. 
\begin{verbatim}
 cd $HOME
 mkdir scratch0/w1
 ln -s ~/scratch0/w1 w1 
 cd w1
 fs setacl . system:anyuser rl
\end{verbatim}
and the SixDesk environment is created by 
\begin{verbatim}
 cd w1
 svn checkout https://svn.cern.ch/reps/sixdesk
\end{verbatim}
or
\begin{verbatim}
 cd w1
 gtar xvzf /afs/cern.ch/user/f/frs/public_html/\
SixTrack_run_environment/SixDesk_run_environment.pro.tgz
\end{verbatim}
where the first method will get the latest SVN committed version
and the second method uses a link to a tar file which will be updated 
in parallel with the SVN commits.  

This will create a directory subtree starting with
the directory {\tt sixjobs}. An overview of this directory is
shown in Appendix \ref{sec:commands}. 
The latest source and postscript of this document can be found 
in the directory {\tt sixjobs/doc}.
All the scripts are stored in {\tt sixjobs} itself
and the user should always do a
{\tt cd w1/sixjobs} before executing any commands. It may be
necessary to prefix any typed commands with a {\tt ./} if your 
shell PATH does not include the current directory {\tt "."}. 
(The source version of each script can be found
in {\tt sixjobs/scripts} as will be discussed later.)

\section{Setting up a Study}
\label{sec:study}
At this point it is important to check that the LOCALE environment variable
LANG has the value C. If not, {\tt setenv LOCALE "C"} in the {\tt csh}.
It has been noted that the {\tt awk} program gives errors if this is not the
case. (This is normally set, but it it is not the default on some laptops
and on MacOS.)
To facilitate the description of the various procedures we use this workspace
{\tt w1} and a study {\tt job\_tracking} as an illustration using the mask
file {\tt job\_tracking.mask} included in the release in the {\tt sixjobs/mask}
directory. Note that the name of the
study is contained in the variable {\tt LHCDescrip} i.e. if we have a study
called {\tt lhc\_nob1} then we must define it 
with {\tt export LHCDescrip=lhc\_nob1}. It should also be noted that amplitudes
and amplitude ranges are specified in beam $\sigma$.
There are several supported types of study, short studies using SixTrack and possibly
Sussix, long tracking studies using SixTrack, and DA map production using the SixTrack DA version.
Performing a tracking study involves running many jobs for an LHC configuration with
many different initial phase space amplitudes and angles and linear tunes. The LHC configuration is
defined by a mask file in the {\tt sixjobs/mask} sub-directory and the initial
conditions in the {\tt sixdeskenv, sysenv} files.

The first step is to edit the {\tt sixdeskenv} script and, if necessary, 
the {\tt sysenv} script.

NOTA BENE:It is essential to issue
a {\tt set\_env} command after any modification to these files so that 
they are saved in the {\tt studies/study} directory as described later.
Note also, that as shown below all values must be exported.

\begin{description}
\item [export LHCDescrip=job\_tracking] the name of the study and of the mask file
\item [export basedir=/afs/cern.ch/user/\$initial/\$LOGNAME]
\item [export scratchdir=/afs/cern.ch/user/\$initial/\$LOGNAME/scratch0]
\item [export trackdir=\$scratchdir/\$workspace]
(can be re-defined to use an area shared between studies or workspaces).
\item [export sixtrack\_input=\$scratchdir/sixtrack\_input/\$workspace/\$LHCDescrip]
(can be changed to point to, and use, an existing set of {\tt sixtrack\_input} files).
\end{description}
The above defaults, apart from the name of the study,
are usually satisfactory but, once chosen,
cannot be changed easily. Other parameters of the run can be decided later.

At this point:
\begin{verbatim}
 cd ~/w1/sixjobs
 set_env
\end{verbatim}
will create the full directory structure of the environment.
As already mentioned this command should also be used to save
modified versions of the {\tt sixdeskenv, sysenv} files before doing
anything else. The command also reports and logs changes. 

While it is recommended to run one study per workspace, most users have
multiple studies in the workspace. The command {\tt ls studies} can be used
to list them and {\tt set\_env "name of the study"} to switch between studies.
This switch copies the {\tt sixdeskenv, sysenv} files from the
{\tt studies} directory to {\tt sixdeskhome} i.e. the current {\tt sixjobs} directory. 

An important new feature is that it is possible to execute many commands
on a specific study without switching. This facilitates running multiple
studies in the same workspace and using commands in batch jobs.  
Thus, instead of switching, a command can be suffixed with a
"study" and optionally a "platform". Examples are shown later.
This option is NOT available for the backup or recall, study or workspace, 
commands described in section \ref{sec:backup}where a {\tt set\_env "study"}
must first be performed.

The {\tt print\_env} command reports the complete environment to the screen;
the command {\tt mywhich} reports a few important values.

Finally note that it is vital that the same versions of SixTrack and madX are used for
all cases in a study. Ideally the madX {\tt run\_mad6t} jobs should be run on the
same type of computer to avoid small numeric differences. Compatible versions of these
programs for LSF and BOINC are specified
in {\tt sysenv}, the so-called "pro" versions. Any changes will
be notified.

\section{Overview of the Data Structure}
The {\tt set\_env} will have created the {\tt sixtrack\_input},
{\tt studies/study},{\tt track} and {\tt work} directories, as well
as various logfile links and directories. 
The files {\tt sixtrack\_input, study, track, work} are in fact
links to the actual directories. An empty file with the name of the study
is also created to facilitate SHELL command completion.     

The {\tt sixtrack\_input} directory will later (after a successful mad6t run)
hold all the SixTrack input 
mother files {\tt fort.3.mother1} and {\tt fort.3.mother2} derived from the mother files
in the {\tt control\_files} directory as well as a {\tt fort.3.mad} and a {\tt fort.3.aux}
and one {\tt fort.2\_"seedno".gz}, {\tt fort.8\_"seedno".gz}, 
{\tt fort.16\_"seedno".gz}
for each seed in the range {\tt istamad} to {\tt iendmad} as defined in
{\tt sixdeskenv}. In addition it will
hold one {\tt mad.dorun\_mad6t*} directory for each {\tt run\_mad6t} command,
with one mad6t LSF job {\tt mad6t\_"seedno".lsf}, one input file {\tt "study"."seedno"}, 
one LSF job log 
{\tt "study"\_mad6t\_"seedno".log} 
and one mad6t output file 
{\tt "study".out."seedno"} for each seed. 

The {\tt studies} directory will hold one directory for each study, in turn
containing the {\tt sixdeskenv, sysenv} files for that study.

The {\tt track} directory will become a hierarchy containing all the SixTrack data
and results as follows:
\begin{description}
\item [Level 1 Seed] typically 1 to 60
\item[         general\_input] containing normalised emittance and gamma
\item [Level 2 simul] for long tracking and/or 
\item [        trans momen] for short/sussix runs
\item [Level 3 tunex\_tuney] e.g. 64.31\_59.32
\item [Level 4 amplitude range] e.g. 18\_20 and 18-22 after post-processing
\item [Level 5 turns exponent] e.g. e5 for $10^5$ turns
\item [Level 6 the phase space angle] e.g. 67.5  
\item [Level 7] input(links), fort.2,3,8,16.gz and the result fort.10.gz 
as well as the LSF jobs and logs
\end{description}
A partial view of the {\tt track} tree is shown in Appendix \ref{sec:plot}.
A typical lowest level structure (after the study has been completed),
for workspace w1, study job\_tracking, long run (simul), seed 1,
tunes 64.28\_59.31, amplitude range 10\_12, $10^5$ turns, phase space angle 1.5, is:
\begin{verbatim}
 ~/w1/sixjobs/track/1/simul/64.28_59.31/10_12/e5/1.5
 fort.10.gz
 fort.16.gz -> 
 ~mcintosh/scratch0/sixtrack_input/w1/job_tracking/fort.16_1.gz
 fort.2.gz -> 
 ~mcintosh/scratch0/sixtrack_input/w1/job_tracking/fort.2_1.gz
 fort.3.gz
 fort.8.gz -> 
 ~mcintosh/scratch0/sixtrack_input/w1/job_tracking/fort.8_1.gz
 job_tracking%1%s%64.28_59.31%10_12%5%1.5.log
 job_tracking%1%s%64.28_59.31%10_12%5%1.5.lsf
\end{verbatim}
Note that input files are in fact links to sixtrack\_input, in order to save disk space,
apart from the fort.3.gz which is of course different for each job. The result
file is fort.10.gz described in section \ref{sec:fort10}. The .lsf and .log
files are the LSF job file and the LSF job log file for the particular case. 

It should be noted that there is a one to one mapping between the name of a case
and the directory where the input and output files are stored. The general form of
the name of a case in a {\tt long} run is 
\begin{verbatim}
study%seed%s%tune_range%amplitude_range%turns_exponent%angle
\end{verbatim}
In the workspace this case would be found in the directory
\begin{verbatim}
track/'seed'/simul/'tune_range'/'amplitude_range'/  \
e'turns_exponent'/'angle'
\end{verbatim}
To give a specific example, a case named
\begin{verbatim}
job_tracking%1%s%62.31_60.32%10_12%5%18
\end{verbatim} would be found in the {\tt w1/sixjobs/track} directory for 
this study in the sub-directory
\begin{verbatim}
sixjobs/track/1/simul/62.31_60.32/10_12/e5/18
\end{verbatim}
this particular case being for seed number 1, tunex and tuney 62.31 and 60.32,
amplitude range 10 to 12, $10^5$ turns, angle 18. (The directory simul could also
be instead trans or momen for short/sussix runs and the letter 's' in the
name being replaced by 't' or 'm'.)
 
This case name is used in the database, to name the LSF jobs and logs, and is
also included in the SixTrack {\tt fort.6} output. Note that as warned in a
comment in the {\tt sixdeskenv} file this name must
NOT have a \% character nor two consecutive underscores.

The {\tt work} directory will contain the database flat files for reporting
and managing all the tasks and jobs of a particular study.
The most interesting files are {\tt work/taskids}, {\tt work/completed\_cases},
{\tt work/incomplete\_cases} and the sub-directory {\tt work/lsfjobs (or boincjobs)}.
The master file is called {\tt taskids} and contains one line for every case of
a study with the case name followed by one or more LSF or BOINC taskids. The {\tt run\_status}
command reports on the status of the study as described in section \ref{sec:status}.

There are also the other directories:
\begin{description}
\item [bin] containing links to various utility programs used by {\tt run\_six}
({\tt dalie4 dalie6 readda}), {\tt run\_join10} ({\tt joinf10})
and {\tt run\_post} ({\tt read10b})
\item [control\_files] the SixTrack mother files for the collision/injection/beam 2
cases which are used to generate the SixTrack input files
\item [doc] The Latex source and Postscript of this document
\item [inc] other mask files and the prepare\_fort.3 script
\item [mask] the LHC description mask files for madX 
\item [plot] the plotting mask files and scripts in the sub-directory {\tt plot/inc}.
The sub-directory {\tt plot/"study"} will contain a hierarchy matching the {\tt track}
hierarchy, but with the {\tt angle} directory replaced by a hidden directory {\tt .angle}  
containing some or all of the files {\tt averem.eps.gz distance.eps.gz fort.10.gz 
fort.15.gz fort.30.gz maxslope.eps.gz smear.eps.gz survival.eps.gz test.ps.gz}.
\item [sixdeskTaskIds] the study TaskIds for BOINC
\item [scripts] the source and copies of the commands
\item [utilities] the various LSF job masks for {\tt run\_mad6t} and {\tt run\_six}  
\end{description}
\section{Running mad6t to produce the basic SixTrack input files}
\label{sec:mad_6t}
NOTA BENE: It is essential to have a \$LHCDescrip.mask file
in the subdirectory {\tt mask} in order to run madX to produce
the SixTrack input files. This mask file in turn references many
LHC Database files and this often requires some checking.
Sample mask files can be found in the directory mask:
\begin{verbatim}
 jobref503_withbb_coll.mask  jobrefslhc_withbb_coll.mask  
 jobref503_withbb_inj.mask   jobref503_inj.mask        
 jobrefslhc_inj.mask         jobrefslhc_withbb_inj.mask
 jobref503_coll.mask         jobrefslhc_coll.mask        
 job_tracking.mask
\end{verbatim}
The file job\_tracking.mask is copied from
/afs/cern.ch/eng/lhc/optics/SLHCV3.01 and all the
others from /afs/cern.ch/eng/lhc/optics/SLHCV2.0 \ref{}.

In order to investigate a
series of random seeds the particular seed number in the MAD input
file has to be replaced by a variable name:
\begin{verbatim}
Set, SEEDSYS , 1 ;
Set, SEEDRAN , 1 ;
\end{verbatim}
becomes
\begin{verbatim}
Set, SEEDSYS , %SEEDSYS ;
Set, SEEDRAN , %SEEDRAN ;
\end{verbatim}
The place keepers {\tt \%SEEDSYS} and {\tt \%SEEDRAN} will be replaced
automatically by a proper seed number by {\tt run\_mad6t} based on the
current seed number based on the {\tt istamad and iendmad} variables
defined in {\tt sixdeskenv}. 

In addition, in order to use the correct bunch size, the place keeper 
{\tt \%NPART} is replaced by the value specified for the {\tt bunch\_charge}
 variable also in {\tt sixdeskenv}.

Please note that the MAD-X {\tt sixtrack} command takes its
information from the last {\tt Twiss, save} commmand. It is a
sensible precaution to put these commands consecutively in the MAD
script and mask files.

The version of madX to be used for the conversion runs is defined
in {\tt sysenv} and defaults to the current production version.
At this point it is also necessary to specify in {\tt sixdeskenv}
\begin{description}
\item [pmass=938.272046] The mass of the proton \cite{NIST} which can be reset to
938.271998 for backwards compatibility with earlier studies.
\item [bunch\_charge=1.1500e+11] New bunch\_charge variable for fort.3.mother1\_[col/inj]
\item [runtype=] inj or coll for injection/collision for an LHC lattice.
For more information on how to define runtypes
for either the LHC or other machines please see section \ref{sec:otherlattices}.
\item [beam=] null or b1 or B1 for Beam1, b2 or B2 for Beam2.
\item [CORR\_TEST=]0 or 1 if check\_mad6t is to copy the corrector strengths
for each seed into one file in sixtrack\_input.
\item [fort\_34=] If null the fort.34 files will not be copied 
to the {\tt sixtrack\_input} directory. These files define the multipole strengths for the
linear lattice when doing Second Order Detuning and Distortion (SODD)
 analysis in expert mode.
\item [istamad=1] first seed for madX.
\item [iendmad=] normally 60 (maximum 64) but it is recommended to use 
iendmad=1 until the results of the {\tt run\_mad6t} are considered satisfactory.
The seeds are used to generate variations of the basic LHC machine with different
randomly generated magnet errors.
\item[madclass=] 8nm for 8 normalised CPU minutes, or 1nh for 1 normalised hour
for the {\tt run\_mad6t} LSF jobs where 8nm is often enough but 1nh may be
necessary.
\end{description}
After a {\tt set\_env}, a {\tt run\_mad6t} or {\tt run\_mad6t -i} may be performed.
The -i option means run interactively and get the output to the screen or
redirected to a file and can be useful for testing the mask file and mad6t run.
The {\tt platform} option is ignored as mad6t runs are either performed on 
the desktop or on LSF. For reasons of numerical compatibility all the {mad6t}  
runs should be performed on the same type of machine.
Subsequently, LSF jobs may be used by {run\_mad6t}, one per seed, and they will
run in parallel. 
In the case of multiple studies in a workspace, either a {\tt set\_env "study"} 
can be performed or the study can be specified on the command as 
{\tt run\_mad6t} "study".
In all cases, the success/failure/correctness of these runs should
be verified. The script {\tt check\_mad6t} checks what it can but it is
essential to have a look at the madX output. This output can be found in the
most recent {\tt sixtrack\_input/mad.run\_mad6t*} directory.
Once the madX run has completed successfully for one seed, {\tt iendmad}
can be set to the desired value (typically 60/64) and a new {\tt run\_mad6t} command
performed. Every {\tt run\_six}, see later, starts with an internal {\tt check\_mad6t}.
\section{Problems, cleaning up, and support} 
There are often problems at this initial stage: at any time the command
{\tt rm -r work/* track/* sixtrack\_input/*} will completely clean up
and allow a restart from the beginning. Help 
and diagnostics and error messages will be found
in the {\tt sixdesk.log} file, one per study \footnotemark.
\footnotetext {Help with these procedures is
always available from Eric.McIntosh@cern.ch,
by telephone, or by SKYPE to mcintosh94. 
Accelerator Physics issues
should be initially addressed to Massimo.Giovannozzi@cern.ch or
Riccardo.de.Maria@cern.ch.}

\section{run\_six - Launching SixTrack Runs}
%
\label{sec:run_six}
It is usual to commence a study with some short runs or a 
few long runs using LSF, perhaps with just a few seeds.

The script {\tt run\_six} is used after setting the various
variables as explained below.
It will automatically launch tracking jobs into the LSF batch
system using the batch scripts in the directory {\tt utilities}
or into the BOINC job submission buffer.
It is recommended to use LSF batch for short runs and for exploratory studies,
or when the binary files are required for detailed examination.
BOINC is recommended for production runs and for large studies of more than 
a few thousand jobs. Note that BOINC does not produce the fort.20 graphics nor
return any binary files. Only the {\tt fort.10.gz}  result file is returned which is
enough for the subsequent post-processing. 
In addition note that the platforms LSF and BOINC should
not both be used in a single study; rather, once the preliminary investigations
are complete using LSF, a new study can be cloned to run full scale
production with BOINC using the same {\tt sixtrack\_input files} thus
avoiding another redundant {\tt run\_mad6t}. 
The experienced user is free to modify the script as explained later.
The following variables, all exported, as specified in the {\tt sixdeskenv} file
are used by this and subsequent scripts.

\begin{description}
\item [tunex \& tuney] The required horizontal and vertical tunes.
These may be different for collision and injection as shown later.
\item [emit] The normalised LHC emittance.
\item [e0] Energy of reference particle depending on runtype
\item [dpini \& dpmax]
  At injection the initial relative momentum deviation is set to
  `dpini=0.00075' and at top energy it is set to `dpini=0.00027'.
  For the determination of the (non-linear) chromaticity a 
  wider range is used: `dpmax=0.002' (see below).
\item [kstep]
  Used to define the step width of the phase space angle. For
  further information see section \ref{sec:run_six:shortrun}. The
  phase space angle is related
  to the emittance ratio via 
  $\phi=\arctan\left(\sqrt{{\epsilon_y}/{\epsilon_x}}\right)$,
  where the emittance is defined as
  $\epsilon_z=A_z*A_z/\beta_z$,
  for z=x,y.
\item [dimen]
  The dimensionality of phase space can be chosen between 4 and 6. In
  the latter case the full six--dimensional tracking is done including
  cavities.
\item [chrom]
  To correct for slight differences between MAD and SixTrack the
  chromaticity is routinely corrected by setting `chrom' to 1 and using
\item[chrom\_eps=0.0000001] This operation will not be performed for 
`chrom=0' but {\tt chromx chromy} will be used instead. 
\item[chromx=2.] and
\item[chromy=2.] being the values used when `chrom=0'.

\item [sussix]
  To determine precise values for the detuning calculation this switch
  should be set to: `sussix=1'. It uses the sussix
  program~\cite{lines3}. This option is only valid for the short run
  configuration (see section \ref{sec:run_six:shortrun}).
  (Problems have been found when using {\tt sussix} in the 6D case.)
\end{description}

{\tt run\_six} can handle three different modes of tracking:
Normally initial investigations are carried out with {\tt short}
runs (and possibly {\tt sussix}).

\begin{enumerate}
\item Short run ---
  This run mode is used to find chromaticity and detuning as a
  function of $\delta$ and amplitude respectively. Typically this is
  done with just 1,000 turns (activate with {\tt short=1} in {\tt sixdeskenv}).
  The other variables in {\tt sixdeskenv} for this run are described in 
  section \ref{sec:run_six:shortrun}.
\item Long run ---
  This mode is meant for the dynamic aperture determination
  proper (activate with {\tt long=1} in {\tt sixdeskenv}).
  The other variables for this run are described in section \ref{sec:run_six:longrun}.
\item Differential Algebra (DA) run ---
  If high order Taylor maps are needed this is the mode to use
  (activate with {\tt da=1}). This mode is mostly for expert use. 
  The {\tt SixTrack} author F. Schmidt should be consulted on how to make best use of it.
  In this case {\tt run\_six} calls the porgrams readda, dalie4 and dalie6
  in the directory {\tt bin}.  
\end{enumerate}

Note that only one type of run may be chosen at any one time; one and only
one of the {\tt sixdeskenv} variables {\tt short, long, da} may be set to 1.

The short/long runs both use seeds as specified by
\begin{description}
\item[ista=\$istamad] Start seed
\item[iend=\$iendmad] End seed
\end{description}
As shown above, the default seed range {\tt ista, iend} for {\tt run\_six} is 
the same as that used for {\tt run\_mad6t}, namely {\tt istamad, iendmad}. 
These values may be changed at any time, for example to submit jobs for a limited
range of seeds, but must clearly be a subset of the {run\_mad6t} values.

\subsection{Short Run}
\label{sec:run_six:shortrun}

\begin{description}
\item[ns1s \& ns2s]
  Lower and upper amplitude range in beam $\sigma$.
\item[nss]
  Amplitude step in beam $\sigma$.
\item[turnss]
  Number of turns which is usually set to 1,000 in this mode.
\item[turnsse]
  This variable should be set to the number of zeros of `turnss',
  i.e. `3' in our example, it becomes part of the data directory
  structure. Therefore, if one decides to redo this analysis at say
  10,000 turns one specifies `turnsse=4' and subsequently the data
  are stored separate from those produced with `turnsse=3'.
\item[writebins]
  This defines after how many turns data are written to output
  files. In this mode it should always be set to: `writebins=1'
  since all turns are needed to find the tunes as a function of
  amplitude.
\item[kini \& kend]
  Initial and end angle in phase space. Typically set from `1' to
  `kmax=5' (see next variable). By specifying `kini=0' the nonlinear
  chromaticity is calculated as well (which uses the `dpmax' setting) and thereafter the initial
  angle is set back to: `kini=1'. Note that the variation from
  `kini' to `kend' is done in steps defined by `kstep'.
\item[kmax]
  This defines the number of phase space angles, e.g. `kmax=5'
  means that each step {\tt kstep} is of: $90^\circ/(kmax+1)=15^\circ$.
\item[reson=0] switch for Guignard resonance calculation
\end{description}

\subsection{Long Run}
\label{sec:run_six:longrun}
\begin{description}
\item [ns1l \& ns2l]
  Lower and upper amplitude range in beam $\sigma$. This range is
  sub--divided into ranges of {\tt nsincl} $\sigma$. In each job 30 pairs of
  particles are evenly distributed in each subrange. The
  close--by pairs are used to find the onset of chaos. Typically we
  find that a variation 2 $\sigma$ is sufficiently dense to find the
  minimum dynamic aperture with a precision of 0.5 $\sigma$.
\item [nsincl] 2 $\sigma$ is standard. A smaller step, of 0.5 say, 
can give better results.
\item [turnsl]
  For the long term tracking we usually track for 100,000 turns or for
one miliion.
\item [turnsle]
  This variable should be set to the number of zeros of `turnsl',
  i.e. `5' in our example. 
\item [writebinl]
  This defines after how many turns data are written to output
  files.

  \emph{Important}: make sure that {\tt writebinl} is large enough
  otherwise huge amounts of data will be created. Occasionally that
  may be of use, however in most cases make sure that no more than a
  total of 1,000 turns are recorded. This implies that for
  `turnsl=100000' the variable should be set to at least `writebinl=100'.
  When running on BOINC, rather than LSF, the binary files are not returned
  but SixTrack is checkpointed every {\tt writebinl} turns. 
  It is recommended to set {\tt writebinl} to 10000 for BOINC runs 
of 100,000 turns or more.
\item[kinil \& kendl]
  Initial and end angle in phase space. As in the `short run' mode
  the variation from `kinil' to `kendl' is done in steps defined by
  `kstep'. 
\item [kmaxl]
  This defines the number of phase space angles, e.g. `kmaxl=5'
  means that each steps amounts to: $90^\circ/(kmaxl+1)=15^\circ$.
  Thus the actual angles are computed by dividing 90
  by kmaxl+1, so 5, 19, 59 for example are reasonable choices.
  The choice of {\tt kmaxl} is discussed in Ref.~\cite{fandmpaper}.
\end{description}

Now the other physics and system parameters must be defined in the
{\tt sixdeskenv} file if the defaults are not suitable.
When the platform is defined as LSF, 
LSF job class definitions will be required:
\begin{description}
\item[platform=LSF] or may be set to BOINC.
\item[longlsfq=1nd] sufficient for 100,000 turns, 60 particles
\item[classs=sixmedium] for short runs.
\item[classda=sixda] for the sixda jobs requiring large memory
\item[sixdeskforce=0] Should normally be left at 0 but may be set to 1 or 2 (see later)
\end{description}
The LSF job class defaults are normally satisfactory, but {\tt longlsfq}
should be set to {\tt 2nd} or to {\tt 1nw} if performing more than $10^5$ turns.

\begin{description}
\item[ibtype=0] or 1 to use the Erskine/McIntosh optimised error function
of a complex number
\item[idfor=1] the closed orbit is added, if set to 0 the initial co-ordinates are unchanged
\item[sixdeskpairs=30] Normal value for 60 particles, naximum of 32. 
\end{description}

Then we have, depending on the {\tt runtyp}:
\begin{verbatim}
if test $runtype = "inj"
then
\end{verbatim}
\begin{description}
\item[e0=450000.] (energy)
\item[gamma=479.6] (gamma)
\item[dpini=0.00075] (initial relative momentum deviation)
\end{description}
\begin{verbatim}
elif test $runtype = "col"
then
\end{verbatim}
\begin{description}
\item[e0=7000000.]
\item[gamma=7460.5]
\item[dpini=0.00027]
\end{description}
\begin{verbatim}
fi
\end{verbatim}
\begin{description}
\item[dpmax=0.002] maximum momentum deviation for a short term run
\end{description}

Next we have the tunes again depending on the runtype:
\begin{description} 
\item[tune=0]
\end{description} 
In this case the {\tt run\_six} will make a special local LSF run
to compute the tunes. Alternatively the tunes may be specified
and in particular a tunescan can be performed where 
the tunes will be computed on a straight line from
(tunex,tuney) with gradient deltay/deltax up to and including (tunex1,tuney1).
The tunes must be 10 .le. tune .lt. 100 in format dd.dd[d][d].
The folowing example specifies the tunes (64.28,59.31) with no scan
(because {\tt tunex = tunex1, tuney = tuney1}. 
\begin{verbatim}
if test $runtype = "inj"
then
\end{verbatim}
\begin{description}
\item[tunex=64.28] Start value
\item[tuney=59.31] Start value
\item[deltax=0.001] Increment to tunex
\item[deltay=0.001] Increment to tuney
\item[tunex1=64.28] End value
\item[tuney1=59.31] End value
\end{description}
\begin{verbatim}
elif test $runtype = "col"
then
\end{verbatim}
\begin{description}
\item[tunex=64.31]
\item[tuney=59.32]
\item[deltax=0.001]
\item[deltay=0.001]
\item[tunex1=64.31]
\item[tuney1=59.32]
\end{description}
\begin{verbatim}
fi
\end{verbatim}
The total number of jobs/cases can be computed as the product of the number of
seeds, the number of tune pairs, the number of amplitude intervals 
and the number of angles.
The total number of cases and progress is reported in the work directory and
can be examined with the {\tt run\_query} or {\tt run\_status} commands. 

Each batch job returns a result file {\tt fort.10.gz} to the {\tt track}
tree/hierarchy. For LSF studies, if the CASTOR switch is on in the {\tt sysenv} file,
all the result files, including the binary files and the fort.6 output, 
are compressed in a tar file and written to 
{\tt \$CASTOR\_HOME/direct\_track/*} where the 
{\tt direct\_track} tree matches the {\tt track} tree in AFS.

It is not recommended to run a study of more than 30,000 jobs/cases in a
single workspace but up to 100,000 still works. A study can be split 
over two or more workspaces, perhaps by seed number, and the results combined.

Problems may, indeed often, arise when it is necessary to run the script more than once,
either because of a system crash, batch daemon not responding, jobs lost, or some other error.
The script {\tt run\_six} can be rerun as often as necessary, but by default will not re-submit
jobs. The {\tt run\_six} can either be submitted as a batch job and/or with a different seed range
({\tt ista, iend}). 
The {\tt run\_six} script also keeps a copy of each LSF job in the {\tt track}
hierarchy along with the data files or links 
{\tt fort.2.gz fort.3.gz fort.8.gz fort.16.gz}. 
It also updates the {\tt taskids file in the}{\tt sixjobs/work} directory.
The {\tt taskids} file contains one line for each case. Each line
contains the JobName (which can be mapped to the case directory) and the
associated LSF job\_ID(s) or BOINC ID(s). 
If {\tt run\_six} is called more than once, it ignores cases
where a non-zero fort.10.gz exists, but otherwise deletes the tracking input files and 
re-generates them. Even simpler recovery is available with the 
{\tt run\_missing\_jobs} command.
\emph{It is essential to wait for existing LSF jobs to complete, or to cancel them, 
and do at least two {\tt run\_status} commands before invoking {\tt run\_missing\_jobs}.}
This procedure has proven to be extremely effective.
In the worst of all cases wait for, or cancel, outstanding batch jobs 
with the {\tt bkill 0} command, 
and delete everything except the {\tt sixtrack\_input}  with a
{\tt rm track/* work/*} command and restart.  

With BOINC the situation is simpler; near the end of a study a {run\_incomplete\_cases}
command will re--submit jobs for those cases where no results are available.
\section{Monitoring the progress of the study.}
\label{sec:status}
The script {\tt run\_status} looks into the database {\tt work} directory and updates and summarises
the status of the study. The {\tt run\_query} command gives a quick look without
any updating. It first counts the number of cases in {\tt work/taskids}.
It reports the number of LSF batch jobs generated, possibly
more than one per case, and does an LSF {\tt bjobs} to report on job status.
Finally it performs a rather time-consuming search to find the number of unfinished cases.
It also produces the files {\tt completed\_jobs}, {\tt incomplete\_jobs}
and possibly {\tt missing\_jobs} in the
{\tt \$sixdeskwork} directory.  When all cases are complete the {run\_join10} procedure can
be initiated. While the {\tt run\_status} command may take some time, many minutes for
50,000 cases, it makes it very easy indeed to recover from LSF failures. For example, in
a recent study with 43,100 jobs over 800 jobs failed. One {\tt run\_missing\_jobs}
command re-submitted them automatically to successfully complete the tracking.  
As an alternative, the {\tt run\_six} command can be re-issued and will not re-submit
jobs if {\tt sixdeskforce = 0}. It will not re-submit jobs which have been completed successfully
if {\tt sixdeskforce = 1} AND a {\tt fort.10.gz} result file has been created.

There are also cases of database corruption. These are cleaned up, after all running LSF jobs
have terminated, by the {\tt correct\_cases} command. Overall the environment has
been proven to be rather robust and almost all errors can be recovered
as long as there is a valid {\tt work/taskids} file.

\section{The fort.10 File}
\label{sec:fort10}
The structure of the {\tt fort.10} files is shown in Tables
\ref{T-PPD} and \ref{T-PPD2}, which have been taken from the official
SixTrack manual \cite{SixTrack} and updated with recent additions.
All values are double-precision numbers, values encoded if necessary.
There is one (very long) line per particle.

\newcounter{dst} \setcounter{dst}{0}

\begin{table}%\small
\caption{Post-processing data of the {\tt fort.10} file}
\vspace{1em}
\label{T-PPD}
\centering
\begin{tabular}{|c|c|}
  \hline
  {\bf \# of Column} & {\bf Description} \\
  \hline \stepcounter{dst}
  \thedst & Maximum turn number \\
  \hline \stepcounter{dst}
  \thedst & Stability Flag (0=stable, 1=lost) \\
  \hline \stepcounter{dst}
  \thedst & Horizontal Tune \\
  \hline \stepcounter{dst}
  \thedst & Vertical Tune \\
  \hline \stepcounter{dst}
  \thedst & Horizontal $\beta$--function \\
  \hline \stepcounter{dst}
  \thedst & Vertical $\beta$--function \\
  \hline \stepcounter{dst}
  \thedst & Horizontal amplitude $1^{st}$ particle\\
  \hline \stepcounter{dst}
  \thedst & Vertical amplitude $1^{st}$ particle\\
  \hline \stepcounter{dst} \thedst & Relative momentum deviation
  \mbox{$ \frac{\Delta p}{p_o}$}\\
  \hline \stepcounter{dst}
  \thedst & Final distance in phase space \\
  \hline \stepcounter{dst}
  \thedst & Maximum slope of distance in phase space \\
  \hline \stepcounter{dst}
  \thedst & Horizontal detuning \\
  \hline \stepcounter{dst}
  \thedst & Spread of horizontal detuning \\
  \hline \stepcounter{dst}
  \thedst & Vertical detuning \\
  \hline \stepcounter{dst}
  \thedst & Spread of vertical detuning \\
  \hline \stepcounter{dst}
  \thedst & Horizontal factor to nearest resonance \\
  \hline \stepcounter{dst}
  \thedst & Vertical factor to nearest resonance \\
  \hline \stepcounter{dst}
  \thedst & Order of nearest resonance \\
  \hline \stepcounter{dst}
  \thedst & Horizontal smear \\
  \hline \stepcounter{dst}
  \thedst & Vertical smear \\
  \hline \stepcounter{dst}
  \thedst & Transverse smear \\
  \hline \stepcounter{dst}
  \thedst & Survived turns $1^{st}$ particle \\
  \hline \stepcounter{dst}
  \thedst & Survived turns $2^{nd}$ particle \\
  \hline \stepcounter{dst}
  \thedst & Starting seed for random generator \\
  \hline \stepcounter{dst}
  \thedst & Synchrotron tune \\
  \hline \stepcounter{dst}
  \thedst & Horizontal amplitude $2^{nd}$ particle\\
  \hline \stepcounter{dst}
  \thedst & Vertical amplitude $2^{nd}$ particle\\
  \hline
\end{tabular}
%\normalsize
\end{table}

\begin{table}%\small
\caption{Post-processing data of the {\tt fort.10} file continued}
\vspace{1em}
\label{T-PPD2}
\centering
\begin{tabular}{|c|c|}
  \hline
  {\bf \# of Column} & {\bf Description} \\
  \hline \stepcounter{dst}
  \thedst & Minimum horizontal amplitude\\
  \hline \stepcounter{dst}
  \thedst & Mean horizontal amplitude\\
  \hline \stepcounter{dst}
  \thedst & Maximum horizontal amplitude\\
  \hline \stepcounter{dst}
  \thedst & Minimum vertical amplitude\\
  \hline \stepcounter{dst}
  \thedst & Mean vertical amplitude\\
  \hline \stepcounter{dst}
  \thedst & Maximum vertical amplitude\\
  \hline \stepcounter{dst}
  \thedst & Minimum horizontal amplitude (linear decoupled)\\
  \hline \stepcounter{dst}
  \thedst & Mean horizontal amplitude (linear decoupled)\\
  \hline \stepcounter{dst}
  \thedst & Maximum horizontal amplitude (linear decoupled)\\
  \hline \stepcounter{dst}
  \thedst & Minimum vertical amplitude (linear decoupled)\\
  \hline \stepcounter{dst}
  \thedst & Mean vertical amplitude (linear decoupled)\\
  \hline \stepcounter{dst}
  \thedst & Maximum vertical amplitude (linear decoupled)\\
  \hline \stepcounter{dst}
  \thedst & Minimum horizontal amplitude (nonlinear decoupled)\\
  \hline \stepcounter{dst}
  \thedst & Mean horizontal amplitude (nonlinear decoupled)\\
  \hline \stepcounter{dst}
  \thedst & Maximum horizontal amplitude (nonlinear decoupled)\\
  \hline \stepcounter{dst}
  \thedst & Minimum vertical amplitude (nonlinear decoupled)\\
  \hline \stepcounter{dst}
  \thedst & Mean vertical amplitude (nonlinear decoupled)\\
  \hline \stepcounter{dst}
  \thedst & Maximum vertical amplitude (nonlinear decoupled)\\
  \hline \stepcounter{dst}
  \thedst & Emittance Mode I\\
  \hline \stepcounter{dst}
  \thedst & Emittance Mode II\\
  \hline \stepcounter{dst}
  \thedst & Secondary horizontal $\beta$--function\\
  \hline \stepcounter{dst}
  \thedst & Secondary vertical $\beta$--function\\
  \hline \stepcounter{dst}
  \thedst & $Q'_x$\\
  \hline \stepcounter{dst}
  \thedst & $Q'_y$\\
  \hline \stepcounter{dst}
  \thedst & SixTrack Version (encoded in double precision)\\
  \hline \stepcounter{dst}
  53 -- 58 &  Closed Orbit\\
  \hline
  59 & The number of the Random Set\\
  \hline
  60 & Tracking CPU time in seconds\\
  \hline
\end{tabular}
%\normalsize
\end{table}

\section{The Other SixTrack Output Files and What the Run Environment
  Does With Them.}
\label{sec:otherfiles}

By default the scripts automatically tar all output
files and store them directly to CASTOR.
In order to view these files they have to be copied from CASTOR using
the standard CASTOR tools like {\tt rfcp}, {\tt rfrm} and 
{\tt nsls, nsrm, nsmkdir} etc. The option {\tt --help} is available
for each of these commands.

Although the full description of all output files is beyond the scope
of this note; an ascii file worth
looking at is the {\tt fort.6.gz} which gives an explicit description of
all operations and possible failures of the SixTrack run as seen by
the program. Also one finds the tracking data in the
binary files {\tt fort.90.gz fort.89.gz etc} down to {\tt fort.59.gz}
in the cases of 64 particles/32 pairs, 
which may be useful for further
analysis. These and other files are all described in the
SixTrack manual \cite{SixTrack}. Note that these files are not available
from BOINC runs which are for full scale production; LSF should be used
for testing and debugging. 

\section{Post-processing with run\_join10}
{\tt run\_join10} gathers the results of completed jobs and
produces combined output files at level 4 in the {\tt track} tree. 
It also deletes any results of previous {\tt run\_join10} commands.
The result directory is named {\tt \$ns1l-\$ns2l} e.g. 14-20 as compared to the
other amplitude directories e.g. 14\_16, 16\_18, etc.
There is one additional parameter for run\_join10 namely
{\tt turnsemax} which is set automatically by {\tt sixdeskenv} 
to \$turnsle or \$turnsse
for long/short studies. Missing files are reported to {\tt sixdesk.log}
but even incomplete results may be useful. This procedure makes use
of the utility program {\tt joinf10} in the directory {\tt bin}.

\section{Post-processing with run\_post}
%
\label{sec:run_post}
After possibly joining the {\tt fort.10} files they can be
postprocessed to find chaotic boundaries and particle losses by using
the script {\tt run\_post}, which in turn uses the program {\tt read10b}
in the directory {\tt bin}.  

There are a couple of options for plotting.
The  {\tt sixdeskenv} variables {\tt iplot} and {\tt kvar} specific to {\tt run\_post} are used.
If {\tt iplot=1} a plot is produced for each seed and the results
can be found in the {\tt plot} directory. Note that setting 
{\tt iplot} to 1 will produce a
possibly huge amount of data, even if it is compressed with gzip.
The variable {\tt kvar} should be set to 1 (the default) to 
obtain the DA as a function of angles for a {\tt long} study,
and the DA over ALL seeds and angles is plotted for each angle
even if {\tt iplot=0}. Note that this is an extremely time consuming
procedure. (For special post-processing of the 
{\tt fort.10.gz} files the set of {\tt fort.15} files are
produced and saved when {\tt iplot = 1})

The following variables are used, almost certainly with the same values
as used by {\tt run\_six} and {\tt run\_join10}:
\begin{description}
\item [kinil \& kendl]
  Initial and end angle in phase space. The variation from `kinil' to
  `kendl' is done in steps defined by `kstep' (see below).
\item [kmaxl]
  This defines the number of phase space angles, e.g. `kmaxl=5'
  means that each step is of: $90^\circ/(kmaxl+1)=15^\circ$.
\item [kstep]
  Used to define the step width of the phase space angle.
\item [Ampl]
  The amplitude range in sigma. To distinguish the original amplitudes
  ranges from the joined ranges
  a hyphen is used `--' instead of an underscore `\_'. {\tt Ampl}
is set automatically using {\tt ns1s/ns2s} or {\tt ns1l/ns2l}.
\item[turnse]
  This variable should be set to the number of zeros of number of
  turns processed,
  i.e. `5' in our example, it is part of the data directory
  structure. 
\item [short]
  In the example `short=0' means that the mode short run is not
  activated.
\item [long]
  In the example `long=1' means that the mode long run is activated.
\item [iplot]
  No plotting for each seed if `iplot=0'. If this flag is set to `1' or '2' the
  following graphics are produced:
  \begin{itemize}
  \item Short run
    \begin{itemize}
    \item Chromaticity, i.e. the tune versus $\delta$.
    \item Horizontal and vertical detuning each in one plot for all
      tracked phase space angles.
    \item Tune foot print, i.e. vertical tune versus horizontal tune
      with the amplitude as a parameter.
    \end{itemize}
  \item Long run
    \begin{itemize}
    \item End value of the distance in phase space `d(turns)' of 2
      initially close--by particles as a function of initial
      amplitude.
    \item Fitted slope of $\log{d(turns)}$ versus $\log{(turns)}$ of
      the distance in phase space of 2 initially close--by particles
      as a function of initial amplitude. For details of the meaning
      of these two chaotic definitions please refer to
      reference~\cite{LHC8}.
    \item Survival plot, i.e. survival time versus initial amplitude.
    \item Horizontal and vertical smear as a function of initial
      amplitude.
    \item Phase space averaged amplitude versus initial amplitude.
    \end{itemize}
  \end{itemize}
  For `iplot=2' these plots are automatically printed using
  your normal Linux \$PRINT\_CMD.
  Obviously, great care has to be taken to avoid a
  swamping the printer. The graphics are stored as files
  like {\tt test.ps.gz} in the directory  
  {\tt sixjobs/plot/"study"} with the same tree structure as the
  {\tt track} tree but with a hidden {\tt .angle} directory.
\end{description}

The DA* result files, one per angle, from {\tt run\_post} in long mode
are called {\tt DAres."Study"."Tunes."Turns exponent"."Angle"},
and are stored in the {\tt sixjobs} directory itself.
The contents and format are defined in the next section \ref{sec:DAres}
It should be noted again that setting {\tt kvar=1}
causes {\tt run\_post} to take a long time, even a very long time of
many hours, for a large number of angles. If both {\tt iplot} and
{\tt kvar} are set to 0 the {\tt DAres.Study.Tunes.Turns exponent.Angle}
files are produced anyway and can be processed with {\tt run\_awk} 
or your own procedures. A partial view of the {\tt sixjobs/plot} 
directory is shown in Appendix C.

\section{The DAres files}
\label{sec:DAres}
These files contain the following columns:
\begin{itemize}
\item Run name for particular seed.
\item ``Strict'' chaotic boundary via slope method~\cite{LHC8}.
\item ``Certain'' chaotic boundary via large distance in phase space
  method~\cite{LHC8}.
\item Dynamic aperture concerning the phase space averaged amplitude
  (preferred value).
\item Raw dynamic aperture concerning initial amplitude (to be used
  with care).
\item Lower bound of tracked amplitude range.
\item Upper bound of tracked amplitude range.
\end{itemize}


{\tt run\_awk} uses the DAres* files from {\tt run\_post}, reports to the screen,
and produces a simple plot file named DAres*.plot.

\section{Running studies in batch and/or parallel}
In summary a study can be run in the following steps:
\begin{enumerate}
\item Create a study by {\tt set\_env} and after creating the
{\tt mask} file in the {\tt mask} directory do a {\tt run\_mad6t}.
Repeat as often as necessary until {\tt check\_mad6t} is successful
and the {\tt mad6t} output has been verified.
\item For a large study use {\tt run\_six} in a batch job by
for example:
\begin{verbatim}
  bsub -q1nd 'cd w1/sixjobs;run_six job_tracking LSF' 
\end{verbatim}
to submit all cases of the study {\tt job\_tracking} in workspace {\tt w1} using LSF.
The progress of the command can be monitored by
{\tt tail -f sixdesk.log} or {\tt bpeek "LSFJobID"}. 
\item Use {\tt run\_status} or {\tt run\_status job\_tracking}, as frequently 
as convenient until all cases are complete or all batch jobs have terminated.
The command {\tt correct\_cases} can be used if the database appears to be
inconsistent (but only after all LSF jobs have terminated). It checks each case and updates 
the complete/incomplete status. 
If there are incomplete cases with LSF do a {\tt run\_missing\_jobs}. With BOINC use
a run\_incomplete\_cases to speed up the completion of the study or to handle the tail
of incomplete cases. When all cases have been completed do a
\item {\tt run\_join10} or {\tt bsub -q1nd 'cd w1/sixjobs;run\_join10 job\_tracking'} 
followed by a
\item {\tt run\_post} or {\tt bsub -q1nd 'cd w1/sixjobs;run\_post job\_tracking'}
\end{enumerate}

\section{Lockfiles}
It is necessary to use locks in order to
avoid conflicting modifications to a file. A directory is locked if
it contains a file sixdesklock in read only mode (444).
If a script fails or dies for some reason or a batch job
is killed this lock file may be left behind. The script
{\tt check\_locks} reports lock status for the current study;
{\tt check\_all\_locks} for the workspace. If the study itself is
locked, it will be reported first as such a lock inhibits further
checking. 
{\tt unlock "directory"} frees the lock. The {\tt unlock\_all} 
unlocks all locks, but is deprecated when there is more than
one active study in the workspace. If in doubt about a lock, 
and the information shown by the check commands, simply do a 
{\tt check\_lock "directory"} or {\tt cat sixdesklock} and
{\tt ls -l sixdesklock} in the 
relevant directory, to see which script, process and machine are 
holding the lock and since when. The general philosophy is to lock
the study so that only one operation at a time is permitted. Most
command wait for the study to be unlocked but the {\tt run\_results}
command for BOINC just exits on the grounds that it will be run periodically,
nornally as an {\tt acrontab} entry.
 
\section{BOINC}
The sixdesk environment was designed to make it as transparent as possible
to use LSF or the 
Berkeley Open Infrastructure for Network Computing BOINC \cite{Boinc}.
which has made over 100,000 PCs available for tracking studies.
There is one additional step required, namely {\tt run\_results}, to
retrieve result files from the BOINC Web server 
and move the {\tt fort.10.gz} to the appropriate directory.
Only this file is returned, no {\tt fort.6} or binary
or graphics files and nothing is written to CASTOR.
To use BOINC the {\tt sixdeskenv} file is modified to
\begin{description}
\item[platform=BOINC]
\end{description}
and that is all.
\begin{enumerate}
\item {\tt run\_mad6t} operates identically as for LSF.
\item {\tt check\_mad6t} as for LSF.
\item {\tt run\_six} as for LSF. 
\item {\tt run\_status} as for LSF. 
\item {\tt run\_results} must now be called to get the fort.10.gz file from the
BOINC server. It can be called regularly, automatically, by using an acrontab
entry in AFS, and there is an example {\tt acrontab.entry} in the {\tt sixjobs}
directory. In addition a {\tt run\_incomplete\_cases} script has been implemented to
allow the use of multiple tasks for a case. This is particularly useful  
towards the end of a run,
when say 90\% of the cases are complete, in order 
to speed up completion of the study.

Unlike LSF which returns all result files directly to the {\tt track} tree
with no checking and over-writing existing results, BOINC 
compares the results file fort.10.gz with any existing result
and reports any differences to {\tt sixdesk.log}.
\item when all cases are complete the {run\_join10} and {\tt run\_post}
procedures are used as with LSF.
\end{enumerate}

\section{Modifying the scripts/commands}
All modifications should be made to the scripts in the directory
{\tt scripts}. The source script beginning with the characters
"my" should be modified when it exists rather than the derived script
e.g. modify {\tt myrun\_six} and NOT {run\_six}. The script {\tt domyseds},
which operates on all the files/scripts in the file {\tt allscripts},
should then be executed to expand macros in scripts named "my*" to
produce the actual command scripts, and to copy  all the commands
to {\tt ../sixjobs}.

Here is a more or less alphabetic list of the "my" scripts:
\begin{itemize}
\item mybackup\_study to CASTOR
\item mybackup\_workspace to CASTOR
\item mycheck\_all\_locks Tries to check for any locks in the workspace
\item mycheck\_lock Checks a specified directory or {\tt sixjobs}
\item mycheck\_locks Tries to check for any locks in the current study
\item mycheck\_mad6t Does minimal checks on the {\tt sixtrack\_input}
\item mycorrect\_cases Re-generates {completed\_cases} and {\tt incomplete\_cases}
using the {\tt taskids} file and checking for the presence of {\tt fort.10.gz}
\item mydelete\_study deletes all or parts of a study based on replies
to prompts
\item mydorun\_mad6t called by run\_mad6t
\item myget\_all\_betavalues Writes all betavalues to STDOUT
\item mymywhich Writes some important variables to STDOUT
\item myrecall\_study from CASTOR
\item myrecall\_workspace from CASTOR
\item myrerun\_all\_cases (Used to rerun\_all cases on BOINC)
\item myrun\_awk Processes all the DAres* files and writes a summary to STDOUT
\item myrun\_incomplete\_cases Re-submits incomplete cases to BOINC
\item myrun\_incomplete\_tasks (Rarely used)
\item myrun\_join10 Joins all {\tt fort.10.gz} files in preparation for {\tt run\_post}
\item myrun\_missing\_jobs Re-submits any LSF jobs which have failed (after {\tt run\_status}
\item myrun\_post Does the post-processing to produce graphics and the DAres* files
\item myrun\_query A quick and brief status report
\item myrun\_results Retrieves BOINC results
\item myrun\_six Submits all LSF jobs or BOINC Work Units (after {run\_mad6t}
\item myrun\_status Reports status and updates the database for the platform LSF
\item myset\_env Used to save the environment after changes to {\tt sixdeskenv sysenv}, 
ito create a study, or to switch studies
\item myunlock Unlocks the specified directory or {\tt sixjobs}
\item myunlock\_all An unlock of all locks of the current study and any other
workspace locks not specific to a study {\tt sixjobs}, {\tt plot} and {\tt sixdeskTaskIds}
\end{itemize}
and here is a list of some other useful scripts/commands without macros. 
\begin{itemize}
\item bresume\_all Resume all suspended batch jobs
\item bstop\_all Stop all batch jobs
\item get\_wpro A sample script for using from acron to get results
\item minav.awk The awk script used by run\_awk
\item print\_env Writes the {\tt sixdeskenv sysenv} variables and values to STDOUT
\item query\_all Does a {\tt run\_status} for each study in a workspace
\item sub\_wpro A sample script for doing a {\tt run\_six} in batch LSF
\end{itemize}
\section{The Subroutines}
These subroutines are all defined in the file mydot\_profile
and may therefore use macros themselves. A {\tt . ./dot\_profile}
statement is issued when starting a script in order to make
them available.
They are called by the corresponding "my" macros as shown.
\begin{itemize}
\item sixdeskmess() mymess {\tt messagelevel} {\tt text}
where the messagelevel is an integer 0, 1 or 2 and the text is 
a doublequoted string. It uses the {\tt sixdeskenv} parameters 
{\tt sixdeskecho} and {\tt sixdesklevel}. 
If {\tt sixdesklevel} =  0 then only basic and error messages are logged.
If {\tt sixdesklevel} = 1 then additional information is logged and if
{\tt sixdesklevel} = 2 then further debug ifnormation is also logged.
If {\tt sixdeskecho}  = "" only minimum information is echoed to STDOUT
and otherwise all messages are also echoed.
\item sixdeskmktmp() mymktmp {\tt name (dir)} makes a uniquely named temporary file in the 
specified directory {\tt dir} or in {\tt .} and returns the full pathname in {\tt \$name} 
\item sixdeskmktmpdir() mymktmpdir {\tt name (dir)} makes a uniquely named temporary directory
in the specified directory {\tt dir} or in {\tt .} and returns the full pathname in {\tt \$name}
\item sixdeskexit() myexit {\tt integer} exits with the code {\tt integer} freeing any locks
\item sixdeskunlock() myunlock {\tt (dir)} unlocks the specified directory or {\tt .}
\item sixdesklock() mylock {\tt (dir)} Locks the specified directory {\tt (dir)} or {\tt .}
\item sixdesktunes() mytunes Converts the tunes and deltas specified in {\tt sixdeskenv}
to integers and logs the values in {\tt sixdesk.log} 
\item sixdeskinttunes() myinttunes Uses the integer tunes to produce values to replace 
placeholders in fort.3
\item sixdeskamps() myamps Generates the equivalent integer values for amplitudes
\item sixdeskrundir() myrundir {\tt name1 name2} Converts the specified run {\tt name1}
to the equivalent unique directory name which is returned in {\tt \$name2} 
\end{itemize}
The scripts:
\begin{itemize}
\item [my]dot\_boinc
\item [my]dot\_bsub
\end{itemize}
are effectively subroutines used by run\_six. 
The scripts 
\begin{itemize}
\item [my]dot\_env
\item [my]dot\_profile
\end{itemize} 
are called by almost every other script to establish the environment
and make the subroutines available.

\section{Backing up the workspace}
\label{sec:backup}
The AFS scratch disks at CERN are not (yet) backed up.
It is recommended that you periodically run a backup so that data can be recovered 
easily in case of a scratch disk failure. It is also recommended to a
{\tt backup\_study [job\_tracking]} when a study {\tt job\_tracking} is complete.
The study may then be deleted safely.

 These backups can be done rather simply in a batch job. First is shown an example of 
backing up your scratch disks and then the rather more sophisticated
facilities available for SixDesk tracking studies and workspaces.

NOTA BENE: all {\tt castor\_backups} save the logs in your {\tt ~/castor/logs}. Although
these are gzipped and might be useful if you forget the contents of a backup
you may also just want to delete them to save space in your HOME directory.
The contents of backups can be always be determined from the backups
themselves. Note also that for all backups, links are NOT followed. Further
a link to a non-existent file is NOT backed up.

 \subsection{Backing up scratch disks}
 Everyone has a CASTOR account and a default {\tt \$CASTOR\_HOME} directory
{\tt /castor/cern.ch/user/\$INITIAL/\$LOGNAME}. All the {\tt ns} commands like
{\tt nsls, nsrm, nsmkdir} etc use this by default. In this example
all the backups are in the CASTOR directory {\tt scratch\_backups}. 
In the examples below {\tt date} is of the form {\tt yyyy-mm-dd} 
(and {\tt mydate} is just {\tt -yyyy-mm-dd}).
The batch job script, which must have execute permission, is as follows:
\begin{verbatim}
#!/bin/sh
# Do a dated (by day) backup of all scratch disks to 
#  $CASTOR_HOME/scratch_backups
# and log the reports in ~/backup_scratch.reports
#
cd $HOME
mydate="-`date -Idate`"
echo "" >> backup_scratch.reports
echo "Scratch backup reports$mydate" >> backup_scratch.reports
echo "Scratch backup reports$mydate"
echo "" >> backup_scratch.reports
for i in `ls -d scratch*`
do
  echo "Backing up $i to scratch_backups/$i$mydate" \
       >> backup_scratch.reports
  echo "Backing up $i to scratch_backups/$i$mydate"
  castor_backup $i scratch_backups/$i$mydate
  nsls -l scratch_backups/$i$mydate >> backup_scratch.reports
  nsls -l scratch_backups/$i$mydate
done
\end{verbatim}

If the job runs to completion (as can be checked in the LSF STDOUT or the 
backup\_scratch\_reports) a lost disk, scratch0 say, can be restored with a
\begin{verbatim}
cd $HOME
castor_recall scratch_backups/scratch0"mydate"/scratch0
\end{verbatim}
where "mydate" identifies the backup to be used. {\tt nsls scratch\_backups} will show
all available backups. Clearly this can also be done in a batch job.
To recall to a different place
\begin{verbatim}
cd $HOME/scratch99
castor_recall scratch_backups/scratch0"date"/scratch0 .
\end{verbatim}
will recall all the files to the directory {\tt scratch99}.

\subsection{Backing up SixDesk workspaces and directories}
 The intended usage is that in order to free up disk space a study can be 
backed up and then deleted. It may later be recalled to exactly the same
workspace or to a new different workspace. The {\tt backup\_workspace}
command might be
used to backup a complete set of studies. A study may be recalled from 
either a workspace backup or a study backup.

 Deleted studies are never backed up; a non-deleted study cannot be recalled
to the same workspace unless it has first been deleted. 
Deleting a study does NOT delete the {\tt sixjobs}
directory and all
{\tt sixdeskenv} and {\tt sysenv} files are kept in the studies directory. All DAres* files
are also preserved.

 All backup/recall commands should be issued from the {\tt sixjobs} directory as usual
except for {\tt recall\_workspace} which should be one level above e.g. {\tt w1}.

 In these examples "date" is of the form ddmmyy.
 So, normally, for example:
\begin{verbatim}
  cd ~w1/sixjobs
  set_env job_tracking
  backup_study
  delete_study
\end{verbatim}
and subsequently
\begin{verbatim}
  cd ~w1/sixjobs
  set_env job_tracking
  recall_study w1%job_tracking%"date"
\end{verbatim}
or to recall to a new different workspace
\begin{verbatim}
  cd ~w1/sixjobs
  set_env job_tracking
  recall_study w1%job_tracking%"date" w99
\end{verbatim}
  
In the latter case the sixdeskenv files are edited to reflect the new
workspace. In all recalls, an 
existing sixjobs directory is never overwritten, but in all cases the 
sixtrack\_input, track, work, and plot data are recalled along with 
logfiles if possible.
 To recall several studies to a new workspace, the first recall will
recall sixjobs as well as the data; afterwards, {\tt cd} to the workspace
to recall other studies.
 As mentioned a study can be similarly recalled from a workspace backup by
\begin{verbatim}
  set_env job_tracking
  recall_study w1%"date" to restore it to w1 
\end{verbatim}
or 
\begin{verbatim}
  set_env job_tracking
  recall_study w1%"date" w99 to a new empty workspace.
\end{verbatim}
 Finally a complete workspace can be recalled to the same workspace or a
different one.
To backup a workspace
\begin{verbatim} 
  cd ~/w1/sixjobs 
  backup_workspace
\end{verbatim}
and to recall
\begin{verbatim}
  cd ~/w1
  recall_workspace w1%"date"
\end{verbatim}
or, since a new different workspace is empty, we must first
get a copy of the {\tt recall\_workspace} script  
\begin{verbatim}
  cd ~/w99
  cp SOMEWHERE/recall_workspace .
  ./recall_workspace w1%"date"
\end{verbatim}

 All backups can be found in \$CASTOR\_HOME/workspace\_backups with names like
{\tt w1\%job\_tracking\%"date"} for a study backup
or
{\tt w1\%"date"} for a workspace backup.
The command {\tt nsls workspace\_backups} will list all available backups.

 All backups are automatically restartable from checkpoints as they may take
some time and are subject to the usual system failures. A file 
{\tt backup\_study.list} or {\tt backup\_workspace.list} is used for restarting
from the point of failure. Thus a backup must be completed before a new
backup is started in the same workspace (or the {\tt file backup\_study.list} or
{\tt backup\_workspace.list} must be deleted). The workspace is locked to ensure
only one backup at a time and no switching of studies.

 These procedures may seem complicated but the simple backup, delete, recall
are easy to perform and the complications are necessary to avoid destroying
data and to recover from system failures. Details of the CASTOR backups/
recalls can be found in the {\tt castor\_backup.log} or {castor\_recall.log}
in the {\tt sixjobs} directory.

\section{Other lattices}
\label{sec:otherlattices}

Clearly any lattice file may be processed by the run environment not
just the LHC. The mask file has to be changed as appropriate.
Other changes are described in the main part of this
note, including aesthetic name changes. Also required are definitions
of the {\tt runtype} (see section \ref{sec:mad_6t}). This variable
and the variable {\tt beam} specify which of the {\tt fort.3.mother1} and {\tt fort.3.mother2}
files from the directory {\tt sixjobs/control\_files}
to use. The value of the {\tt runtype and beam} variables simply specify the suffix
of the files to be used from this directory. For how these files are
constructed the user is refered to the SixTrack manual \cite{SixTrack}.
\section{Acknowledgements}
Thanks to F. Schmidt for the original scripts and many helpful discussions
and much advice; to R. Demaria, M. Giovannozzi and T. ~Risselada
for acting as guinea pigs and for many helpful suggestions.
Finally, thanks to S. Fartoukh for setting up the {\tt mask} files in
the {\tt mask} directory.

\begin{figure*}[tb]
\centering
\includegraphics*[width=200mm]{print.eps}
\caption{Example of fu problems.}
\label{dir}
\end{figure*}

\appendix
\section{An overview of the w1/sixjobs directory}
\label{sec:commands}
See Figure \ref{dir}

\section{A partial view of the track tree}
\label{sec:track}
For the purpose of illustration a very small study with just
two seeds, two tune pairs, two amplitude and two angles is used.
\begin{verbatim}
Level 1
track: 1 2 general_input

Level 2
track/1: simul

Level 3
track/1/simul: 62.311_60.321 62.31_60.32

Level 4
track/1/simul/62.31_60.32: 6-7 6.5_7 6_6.5 \
betavalues mychrom sixdesktunes 

Level 5
track/1/simul/62.31_60.32/6-7: e3

Level 6
track/1/simul/62.31_60.32/6-7/e3: 30 60

Level 7
track/1/simul/62.31_60.32/6-7/e3/30: 
fort.10.gz
track/1/simul/62.31_60.32/6-7/e3/60: 
fort.10.gz

Level 6
track/1/simul/62.31_60.32/6_6.5/e3: 30 60

Level 7
track/1/simul/62.31_60.32/6_6.5/e3/30:
fort.10.gz
fort.16.gz
fort.2.gz
fort.3.gz
fort.8.gz
job_tracking%1%s%62.31_60.32%6_6.5%3%30.log
job_tracking%1%s%62.31_60.32%6_6.5%3%30.lsf

track/1/simul/62.31_60.32/6_6.5/e3/60:
fort.10.gz
fort.16.gz
fort.2.gz
fort.3.gz
fort.8.gz
job_tracking%1%s%62.31_60.32%6_6.5%3%60.log
job_tracking%1%s%62.31_60.32%6_6.5%3%60.lsf
\end{verbatim}
Note the correspondance between the LSF jobname and the name of the
directory containing it, as described in Section \ref{sec:study}.
\section{A partial view of the plot tree after post-processing}
\label{sec:plot}
In this view post-processing was performed with {\tt iplot=1} and {\tt kvar=1}.
Again, a very small study with just
two seeds, two tune pairs, two amplitudes and two angles is used.
\begin{verbatim}
plot: inc job_tracking
job_tracking: 1 2 allseeds.eps.gz

job_tracking/1: simul

job_tracking/1/simul: 62.311_60.321 62.31_60.32

job_tracking/1/simul/62.31_60.32: 6-7

job_tracking/1/simul/62.31_60.32/6-7/e3: .1 .2

job_tracking/1/simul/62.31_60.32/6-7/e3/.1:
averem.eps.gz
distance.eps.gz
fort.10.gz
fort.11
fort.15.gz
fort.28
fort.30.gz
fort.40
kvar.eps.gz
maxslope.eps.gz
smear.eps.gz
survival.eps.gz
test.ps.gz

job_tracking/1/simul/62.31_60.32/6-7/e3/.2:
averem.eps.gz
distance.eps.gz
fort.10.gz
fort.11
fort.15.gz
fort.28
fort.30.gz
fort.40
kvar.eps.gz
maxslope.eps.gz
smear.eps.gz
survival.eps.gz
test.ps.gz
\end{verbatim}
Note the "hidden" directories {\tt .1} and {\tt .2} as mentioned in
Section \ref{sec:run_post}.
\section{Alphabetic list of supported commands and scripts}
\label{sec:commandlist}
\begin{itemize}
\item backup\_study to CASTOR
\item backup\_workspace to CASTOR
\item bresume\_all Resume all suspended batch jobs
\item bstop\_all Stop all batch jobs
\item check\_all\_locks Tries to check for any locks in the workspace
\item check\_lock Checks a specified directory or {\tt sixjobs}
\item check\_locks Tries to check for any locks in the current study
\item check\_mad6t Does minimal checks on the {\tt sixtrack\_input}
\item correct\_cases Re-generates {completed\_cases} and {\tt incomplete\_cases}
\item delete\_study deletes all or parts of a study based on replies to prompts
\item dorun\_mad6t called by run\_mad6t
\item get\_all\_betavalues Writes all betavalues to STDOUT
\item get\_wpro A sample script for using from acron to get results
\item mywhich Writes some important variables to STDOUT
\item print\_env Writes the {\tt sixdeskenv sysenv} variables and values to STDOUT
\item query\_all Does a {\tt run\_status} for each study in a workspace
\item recall\_study from CASTOR
\item recall\_workspace from CASTOR
\item rerun\_all\_cases (Used to rerun\_all cases on BOINC)
\item run\_awk Processes all the DAres* files and writes a summary to STDOUT
\item run\_incomplete\_cases Re-submits incomplete cases to BOINC
\item run\_incomplete\_tasks (Rarely used)
\item run\_join10 Joins all {\tt fort.10.gz} files in preparation for {\tt run\_post}
\item run\_missing\_jobs Re-submits any LSF jobs which have failed (after {\tt run\_status}
\item run\_post Does the post-processing to produce graphics and the DAres* files
\item run\_query A quick and brief status report
\item run\_results Retrieves BOINC results
\item run\_six Submits all LSF jobs or BOINC Work Units (after {run\_mad6t}
\item run\_status Reports status and updates the database for the platform LSF
\item set\_env Used to save the environment after changes to {\tt sixdeskenv sysenv},
 to create a study, or to switch studies
\item sub\_wpro A sample script for doing a {\tt run\_six} in batch LSF
\item unlock Unlocks the specified directory or {\tt sixjobs}
\item unlock\_all An unlock of all locks of the current study and any other 
workspace locks not specific to a study {\tt sixjobs}, 
{\tt plot} and {\tt sixdeskTaskIds}
\end{itemize}
\begin{thebibliography}{99}
%
\bibitem{Runsix} M.~Hayes and F.~Schmidt, ``Run Environment for SixTrack'',
  LHC Project Note 300,\\
  (see also http://wwwslap.cern.ch/frs/SixTrack\_run\_environment/manual.ps).
%
\bibitem{SixTrack} F.~Schmidt, ``SixTrack: Version 3, Single Particle
  Tracking Code Treating Transverse Motion with Synchrotron
  Oscillations in a Symplectic Manner, User's Reference Manua'',
  CERN/SL/94--56 (AP)\\
  (see also http://wwwslap.cern.ch/frs/Documentation/doc.htmlx).
%
\bibitem{hansnote} H.~Grote, ``Statistical significance of dynamic
  aperture calculations'', Beam Physics Note 34.
%
\bibitem{NIST} \htmladdnormallink{http://physics.nist.gov/cuu/Constants/index.html}
  {http://physics.nist.gov/cuu/Constants/index.html}.
%
\bibitem{lines3} R.~Bartolini and F.~Schmidt, ``SUSSIX: A Computer
  Code for Frequency Analysis of Non--Linear Betatron Motion'',
  presented at the workshop ``Nonlinear and Stochastic Beam Dynamics
  in Accelerators -- A Challenge to Theoretical and Computational
  Physics'', L\"uneburg, September 29 -- October 3, 1997, CERN SL/Note
  98--017 (AP),
  \htmladdnormallink{http://wwwslap.cern.ch/frs/report/sussix\_manual\_sl.ps.gz}
  {http://wwwslap.cern.ch/frs/report/sussix\_manual\_sl.ps.gz}.
%
\bibitem{fandmpaper}
M.~Giovannozzi and E.~McIntosh, ``Parameter scans and accuracy estimates
of the dynamical aperture of the CERN LHC'', EPAC'06, June 2006, Edinburgh.
%
\bibitem{SIXTRACK6} M.~B\"oge and F.~Schmidt, ``Data Organisation for
  the LHC Tracking Studies with SIXTRACK'', LHC Project Note 99,\\
  \htmladdnormallink{http://wwwslap.cern.ch/frs/report/lhc$\_$pro$\_$note99.ps.Z}
  {http://wwwslap.cern.ch/frs/report/lhc$\_$pro$\_$note99.ps.Z}.
%
\bibitem{LHC8} M.~B\"oge and F.~Schmidt, ``Estimates for Long--Term
  Stability for the LHC'', LHC Project Report 114, presented in part
  at the Particle Accelerator Conference, Vancouver, 12--16 May,
  (1997), AIP Conference Proceedings 405 (1996),
  \htmladdnormallink{http://wwwslap.cern.ch/frs/report/conj97lhc.ps.Z}
  {http://wwwslap.cern.ch/frs/report/conj97lhc.ps.Z},\\
  the poster version:
  \htmladdnormallink{http://wwwslap.cern.ch/frs/report/conj97post.ps.Z}
  {http://wwwslap.cern.ch/frs/report/conj97post.ps.Z} \\
  and contribution to the workshop on ``New Ideas for Particle
  Accelerators'', Santa Barbara, November 1996.
%
\bibitem{Boinc} ``Berkeley Open Infrastructure for Network Computing'',
   http://boinc.berkeley.edu
\end{thebibliography}

\clearpage

\end{document}
